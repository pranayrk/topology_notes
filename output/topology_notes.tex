\documentclass[12pt,twoside]{report}
\usepackage[utf8]{inputenc}
\usepackage{amsmath,amsfonts,graphicx,amsthm}
\usepackage{epsfig,amssymb}
\usepackage{caption}
\usepackage{fancyhdr}
\usepackage{lipsum}
\usepackage{thmtools}
\usepackage{enumitem}
\usepackage[top=1.25in, bottom=1.25in,left=1.5in, right=1.25in]{geometry}

\newtheorem{defn}{Definition}
\newtheorem{note}{Note}
\newtheorem{notation}[subsection]{Notation}
\newtheorem{thm}{Theorem}
\newtheorem{corollary}[subsection]{Corollary}
\newtheorem{eg}[subsection]{Example}
\newtheorem{lemma}[subsection]{Lemma}
\newtheorem{result}{Result}
\declaretheoremstyle[bodyfont=\normalfont]{normalbody}

\declaretheorem[style=normalbody,name=]{ex}
\newtheorem{remark}[subsection]{Remark}
\newtheorem{prop}[subsection]{Proposition}
\newenvironment*{ans}{\textbf{ans.}\space\em\\}{\par}
\newenvironment*{source}{\hfill\scriptsize\textbf{Source.}\space}{\par}

\declaretheoremstyle[headfont=\normalfont]{normalhead}

\makeatletter
\newcommand*{\rom}[1]{\expandafter\@slowromancap\romannumeral #1@}
\makeatother

\widowpenalties 1 10000
\raggedbottom
\setlength{\parindent}{0pt}
\renewcommand{\chaptername}{}

\setlength{\oddsidemargin}{36pt}
\setlength{\evensidemargin}{36pt}

\begin{document}

\tableofcontents
\newpage
\pagenumbering{arabic}

\chapter{Definitions and Theory}
\line(1,0){360}

\section{Topological Spaces}

\begin{defn}
A \textbf{topology} on a set $X$ is a collection $\tau$ of subsets of $X$ satisfying:
    \begin{enumerate}
        \item $\varnothing, X \in  \tau$
        \item  An intersection of finite subcollections of $\tau$ is in $\tau$
        \item A union of any subcollection of $\tau$ is in $\tau$
    \end{enumerate}
The ordered pair $\left (X, \tau\right )$ is called a \textbf{topological space}.
\end{defn}


\begin{defn}
    Let $\left (X, \tau\right )$ be a topological space. An \textbf{open subset} of $X$ is a member of $\tau$.
\end{defn}

\begin{defn}
Let $\tau$ and $\sigma$ be two topologies on a set $X$. We say that $\tau$ is \textbf{weaker} (or smaller, coarser) than $\sigma$ if $T \subseteq \sigma$. In this case, $\sigma$ is then said to be \textbf{stronger} (or larger, finer) than $\tau$.
\end{defn}

\begin{defn}
Let $X$ be any set. The collection $\tau = P\left (X\right )$ is a topology on $X$ and is called the \textbf{discrete topology} on $X$. Here $\left (X, \tau\right )$ is called the \textbf{discrete topological space}.
\end{defn}

\begin{defn}
Let $X$ be any set. The collection $\tau = \left\{ \varnothing, X \right\}$ is called the \textbf{indiscrete topology} on $X$. Here $\left (X, \tau\right )$ is called the \textbf{indiscrete topology}.
\end{defn}

\begin{defn}
    Let $X$ be any set. The collection $\\tau = \left\{ A \subseteq X  :  X \setminus A \text{ is finite } \right\} \cup \left\{ \varnothing \right\}$ is called the \textbf{co-finite topology}.
\end{defn}

\begin{defn}
    Let $X$ be any set. The collection $\\tau = \left\{ A \subseteq X  :  X \setminus A \text{ is countable } \right\} \cup \left\{ \varnothing \right\}$ is called the \textbf{co-finite topology}.
\end{defn}



\begin{defn}
A topology $\tau$ on a set $X$ is said to be \textbf{metrizable} if there exists a metric $d$ on $X$ such that the topology $\tau_d$ generated by the metric $d$ coincides with $\tau$.
\end{defn}

\begin{defn}
Two metrics defined on a set $X$ are said to be \textbf{equivalent} if they generate the same topology. 
In other words, $d_1$ and $d_2$ are equivalent if the collection of open sets in $\left (X, d_1\right )$ and $\left (X, d_2\right )$ are the same.
\end{defn}

\begin{defn}
The topology generated by the Euclidean metric on $\mathbb{R} ^n$ is called the \textbf{usual topology} on $\mathbb{R} ^n$.
For $Y \subseteq \mathbb{R} ^n$, the topology generated by the Euclidean metric is called the usual topology on $Y$.
\end{defn}

\begin{defn}
By a \textbf{neighbourhood} of a point $x$ in a topological space $\left (X, \tau\right )$, we mean an open set containing $x$.
\end{defn}

\begin{defn}
A subset $A$ of a topological space $\left (X, \tau\right )$ is said to be \textbf{closed} if $X \setminus A$ is open in $X$, that is $X \setminus A \in \tau$
\end{defn}

\begin{thm}
    \begin{itemize}
        \item[]
        \item  $\varnothing$ and $X$ are closed in $X$
        \item An intersection of any collection of closed sets is closed in $X$
        \item A union of a \textit{finite} collection of closed sets in $X$ is closed in $X$
    \end{itemize}
\end{thm}

\begin{defn}
    Let $X$ be a topological space and $A \subseteq X$. A point $x \in X$ is said to be an \textbf{interior point} of $A$ if there exists a neighbourhood $U$ of $X$ such that $U \subseteq A$ (or) if there exists an open set $U$ in $X$ such that $x \in U \subseteq A$. 

The set of all interior points of $A$ is called the \textbf{interior} of $A$ and is denoted $A ^\mathrm{o}$.
\end{defn}

\begin{thm}
    Let $\left (X, d\right )$ be a topological space and $S \subseteq X$. Then $S ^\mathrm{o}$ is the largest open set in $X$ that is contained in $S$.
\end{thm}

\begin{thm}
    Let $X$ be a topological space and $S \subseteq X$ Then $S$ is open if and only if $S = S ^\mathrm{o}$.
\end{thm}

\section{Bases and Subbases}

\begin{defn}
    Let $X$ be a topological space. A collection $\beta$ of open subsets of $X$ is said to be a \textbf{basis} for the topology on $X$ if for every open set $U$ in $X$ and $x \in U$, there exists a $B \in \beta$ such that $x \in B \subseteq U$.

    Members of $\beta$ are called \textbf{basis open sets} corresponding to basis $\beta$.
\end{defn}

\begin{note}
    For any topological space $\left (X, \tau\right )$, $\tau$ is a basis for $\tau$.
\end{note}

\begin{note}
    In $\mathbb{R}$ the set $\left\{ \left (x - \varepsilon, x + \varepsilon\right )  :  x \in \mathbb{R}\text{ and }\varepsilon >0 \right\}$ is a basis for the usual topology.
\end{note}

\begin{note}
    For $\left (\mathbb{R}, \tau\right )$, where $\tau$ is the discrete topology, $\beta = \left\{ \left\{ x \right\}  :  x \in \mathbb{R} \right\}$ is a basis for $\tau$ on $\mathbb{R}$.
\end{note}

\begin{note}
    In $\mathbb{R}$ the set $\left\{ \left (x - \frac{1}{n}, x + \frac{1}{n}\right )  :  x \in \mathbb{R}\text{ and }n \in \mathbb{N} \right\}$ is a basis for the usual topology.
\end{note}

\begin{note}
    In $\mathbb{R}$ the set $\left\{ \left (a, b\right )  :  a,b \in \mathbb{Q}\text{ and }a < b \right\}$ is a basis for the usual topology. This is a countable basis for $\mathbb{R}$ with the usual topology.
\end{note}

\begin{note}
    Let $\left (X, \tau\right )$ be a metrizable space. Then $\beta = \left\{ B\left (x, \varepsilon\right )  :  x \in X, \varepsilon >0 \right\}$ is a basis for $\tau$.
\end{note}

\begin{defn}
    Let $\left (X, \tau\right )$ be a topological space. A collection $S \subseteq \tau$ is said to be a \textbf{subbasis} for the topology $\tau$ if for every open set $U$ in $\tau$ and $x \in U$, there exists a finite subcollection $\left\{ S_1, S_2, ..., S_n \right\}$ in $S$ such that $x \in \displaystyle\bigcap_{i = 1}^n S_i \subseteq U$

    Members of $S$ are called \textbf{subbasis open sets} corresponding to $S$.
\end{defn}

\begin{note}
    For a topological space $\left (X, \tau\right )$ a collection $S$ is a subbasis for the topology $\tau$ if and only if the collection of all finite intersections of members in $S$ forms a basis for the topology $\tau$.
\end{note}

\begin{thm}
    Let $\left (X, \tau\right )$ be a topological space and $\beta$ be a collection of open sets in $\left (X, \tau\right )$.

    Then $\beta$ is a basis for the topology $\tau$ on $X$ if and only if every open set $U$ in $\left (X, \tau\right )$ can be written as a union of members in $\beta$.
\end{thm}

\begin{thm}
    If $X$ is a set, a basis for a topology on $X$ is a collection $\beta$ of subsets of $X$ such that 
    \begin{itemize}
        \item for every $x \in X$ there exists $B \subset \beta$ containing $x$.
        \item if $x \in B_1 \cap B_2$, for some $B_1, B_2 \in \beta$, then there exists $B_3 \in \beta$ such that $x \in B_3 \subseteq B_1 \cap B_2$.
    \end{itemize}
\end{thm}

\begin{defn}
    The topology generated by the basis $\beta = \left\{ \left [a, b\right )  :  a, b \in \mathbb{R}, a < b \right\}$ is known as the \textbf{lower limit topology} on $\mathbb{R}$. 

    The space $\mathbb{R}$ equipped with the lower limit topology is known as the \textbf{Sorgenfrey line} $\mathbb{R}_l$.
\end{defn}

\begin{prop}
    The lower limit topology on $\mathbb{R}$ is strictly finer than the usual topology on $\mathbb{R}$
\end{prop}

\begin{defn}
    Let $K = \left\{ \frac{1}{n}  :  n \in \mathbb{N} \right\}$. Consider the basis \\ $\beta = \left\{ \left (a, b\right ) \setminus K  :  a,b \in \mathbb{Q}, a < b \right\} \cup \left\{ \left (a,b\right )  :  a,b \in \mathbb{R}, a < b \right\}$.  The topology generated by the basis $\beta$ is known as the \textbf{$k$-topology} on $\mathbb{R}$.
\end{defn}

\begin{prop}
    The $k$-topology on $\mathbb{R}$ is strictly finer than the usual topology on $\mathbb{R}$.
\end{prop}

\begin{prop}
    The lower limit topology and the $k$-topology on $\mathbb{R}$ are incomparable.
\end{prop}


\begin{defn}
    Let $X$ be a set. Let $S$ be a collection of subsets of $X$ whose union is in $X$. Any such collection is called a \textbf{subbasis} for a topology $X$. (TODO: Verify)
\end{defn}

\begin{thm}
    Let $S$ be a subbasis for a topology on $X$. Then
    \\ $\tau = \{ U \subseteq X  :  $ for every $ x \in U $ there exists a finite number of members $ S_1, S_2, ..., S_n \in S $ such that $ x \in S_1 \cap S_2 \cap ... \cap S_n \subseteq U \}$ is the topology generated by the subbasis $S$.
\end{thm}

\begin{thm}
    Let $\beta$ be a basis for a topology on $X$. Then the topology generated by $\beta$ is equal to the intersection of all topologies on $X$ that contains $\beta$.
\end{thm}

\begin{thm}
    If $\beta$ is a basis for a topology on a set $X$, then the topology generated by $\beta$ on $\tau$ is the smallest topology on $X$ that contains $\beta$.
\end{thm}

\begin{defn}
    Let $X$ be a set. A function $f \colon \mathbb{N} \to X$ is called a \textbf{sequence} in $X$.
\end{defn}

\begin{defn}
    Let $\left (X, Y\right )$ be a topological space. A sequence $\left (X_n\right )$ is said to \textbf{converge} to $x \in X$ if for every neighbourhood $U$ of $x$, there exists $n_0 \in \mathbb{N}$ such that $x_n \in U \;\forall n \geq n_0$
\end{defn}

\begin{defn}
    A sequence $\left (x_n\right )$ is said to be \textbf{eventually constant} if there exists $n_0 \in \mathbb{N}$ such that $x_n = x_{n_0} \;\forall n \geq n_0$.
\end{defn}

\begin{thm}
    In any topological space, every eventually constant sequence is convergent.
\end{thm}

\begin{defn}
    Let $f \colon \mathbb{N} \to X$ be a sequence in $X$. Then for any strictly increasing function $g \colon \mathbb{N} \to \mathbb{N}$, the composition $f \circ g \colon \mathbb{N} \to X$ is called a \textbf{subsequence} of $f$.
\end{defn}

\begin{thm}
    In a topological space $\left (X, \tau\right )$, every subsequence of a convergent sequence is convergent.
\end{thm}

\begin{defn}
    Let $\mathcal{A}$ be a non-empty set. A relation $\leq$ on a set $A$ is called a \textbf{partial order relation} if the following condition holds for all $\alpha, \beta, \gamma$ in $A$:
    \begin{enumerate}
        \item reflexive: $\alpha \leq \alpha$
        \item anti-symmetric: $\alpha \leq \beta$ and $\beta \leq \alpha$ $\implies$ $\alpha = \beta$
        \item transitive: $\alpha \leq \beta$ and $\beta \leq \gamma$ $\implies$ $\alpha \leq \gamma$
    \end{enumerate}
\end{defn}

\begin{defn}
    A \textbf{directed set} $J$ is a set with a partial order relation $\leq$ such that for each pair $\alpha$ and $\beta$ of $J$, there exists a $Y \in J$ such that $\alpha \leq Y$ and $\beta \leq Y$.
\end{defn}

\begin{defn}
    A \textbf{net} in $X$ is a function $f$ from a directed set $J$ to $X$.
\end{defn}

Note that every sequence is a net.

\begin{defn}
    Let $J$ be a directed set with a partial order relation '$\leq$'. A subset $K$ of $J$ is said to be \textbf{cofinal} in $J$ if for each $\alpha \in J$, there exists $\beta \in K$ such that $\alpha \leq \beta$.
\end{defn}

\begin{prop}
    If $g \colon \mathbb{N} \to \mathbb{N}$ is a strictly increasing function then $g\left (\mathbb{N}\right )$ is cofinal in $\mathbb{N}$.
\end{prop}

\begin{thm}
    If $J$ is a directed set and $K$ is cofinal in $J$, then $K$ is a directed set.
\end{thm}

\begin{defn}
    Let $f \colon J \to X$ be a net in $X$. If $I$ is a directed set and $g \colon I \to J$ such that 
    \begin{itemize}
        \item $i < j \implies g\left (i\right ) < g\left (j\right ) \;\forall i, j \in I$
        \item $g\left (I\right )$ is cofinal in $J$
    \end{itemize}
    Then $f \circ g \colon I \to X$ is called a \textbf{subnet} of $f$.
\end{defn}

\begin{defn}
    The net $\left (X_\alpha\right )_{\alpha \in J}$ is said to \textbf{converge} to a point $x \in X$ if for every neighbourhood $U$ of $x$, there exists $\alpha_0 \in J$ such that $x_\alpha \in U$ for all $\alpha \geq \alpha_0$.
\end{defn}

\begin{defn}
    Let $\left (X, \tau\right )$ be a topological space and $S \subseteq X$. A point $x \in X$ is called a \textbf{closure point} of $S$ if for every neighbourhood $U$ of $x$, we have $U \cap S \neq  \varnothing$.

    THe set of all closure points of $S$ is called the \textbf{closure} of $A$ and is denoted $\text{cl}\left (A\right )$ or $\bar{A}$.
\end{defn}

\begin{defn}
    A point $x \in X$ is said to be a \textbf{limit point} of $S$ if for every neighbourhood of $x$, we have $\left (U \cap S\right ) \setminus \left\{ x \right\} \neq  \varnothing $.

    The set of all limit points is called the \textbf{derived set} and is denoted $S'$
\end{defn}

\begin{thm}
    Let $X$ be a topological space and $S \subseteq X$. Then $\bar{S}$ is the smallest closed set in $X$ that contains $S$.
\end{thm}

\begin{prop}
    A subset of a topological space $X$ is closed if and only if $S = \bar{S}$
\end{prop}

\begin{thm}
    Let $\left (X, \tau\right )$ be a metrizable space and $S \subseteq X$. Then $x \in \bar{S}$ if and only if there exists a sequence $\left\langle x_n\right\rangle $ in $S$ such that $x_n \to x$.
\end{thm}

\begin{thm}
    Let $X$ be a topological space and $S \subseteq X$. Then show that $x \in \bar{S}$ if and only if there exists a net $\left\langle  x_\lambda \right\rangle $ in $S$ such that $ x_\lambda \to x$
\end{thm}

\begin{thm}
    Let $X$ be a topological space and $S \subseteq X$. Then $S ^\mathrm{o}$ is the largest open set contained in $S$.
\end{thm}

\begin{thm}
    Let $X$ be a topological space and $S \subseteq X$. Then $S$ is open if and only if $S = S ^\mathrm{o}$.
\end{thm}

\begin{defn}
    Let $\left (X, \tau_1\right )$ and $\left (Y, \tau_2\right )$ be two topological spaces. The collection $\mathcal{B} = \tau_1 \times \tau_2 = \left\{ u \times v  :  u \in \tau_1, v \in \tau_2 \right\}$ is a basis for a topology on $X \times Y$.
    The topology generated by $\mathcal{B}$ is called the product topology on $X \times Y$.
\end{defn}

\begin{thm}
    Let $\left (X, \tau\right )$ and $\left (Y, \sigma\right )$ be two topological spaces. Let $\mathcal{B}$ be a basis for $\tau$ and $\mathcal{C}$ be a basis for $\sigma$.
    Then the collection $\mathcal{D} = \left\{ U \times V  :  U \in \mathcal{B}, V \in \mathcal{C} \right\}$ is a basis for the product topology on $X \times Y$.
\end{thm}

\end{document}

