\documentclass[14pt,twoside]{extreport}
\usepackage[utf8]{inputenc}
\usepackage{amsmath,amsfonts,graphicx,amsthm}
\usepackage{epsfig,amssymb}
\usepackage{caption}
\usepackage{fancyhdr}
\usepackage{lipsum}
\usepackage{thmtools}
\usepackage{enumitem}
\usepackage{bm}
\usepackage{mathrsfs}
\usepackage[margin=0.75in]{geometry}

\tolerance=1
\emergencystretch=\maxdimen
\hyphenpenalty=10000
\hbadness=10000
\widowpenalties 1 10000
\raggedbottom
\raggedright
\setlength{\parindent}{0pt}
\renewcommand{\chaptername}{}
\newcommand{\hhrule}{\vspace{1cm}\hrule\vspace{1cm}}

\newtheoremstyle{dotless}{}{}{}{}{\bfseries}{}{\newline}{}
\theoremstyle{dotless}
\newtheorem*{defn}{Definition}
\newtheorem*{thm}{Theorem} % Anything that needs to be proved
\newtheorem*{note}{Note} % Anything from the source that provides extra information
\newtheorem*{remark}{Remark} % Comments by self
\newtheorem*{lemma}{Lemma} % Any theorem that does not need to be proved
\newtheorem*{prop}{Proposition} % Anything that comes from another subject and does not need to be proved
\newtheorem*{result}{Result} % Anything from the source that is not a theorem or lemma

\everymath{\displaystyle}

\begin{document}

\newpage
\pagenumbering{arabic}

\chapter*{Topology}
\line(1,0){360}

\begin{defn}
A \textbf{topology} on a set $X$ is a collection $\tau$ of subsets of $X$ satisfying:
    \begin{enumerate}
        \item $\varnothing, X \in  \tau$
        \item  An intersection of finite subcollections of $\tau$ is in $\tau$
        \item A union of any subcollection of $\tau$ is in $\tau$
    \end{enumerate}
The ordered pair $\left (X, \tau\right )$ is called a \textbf{topological space}.
\end{defn}


\begin{defn}
    Let $\left (X, \tau\right )$ be a topological space. An \textbf{open subset} of $X$ is a member of $\tau$.
\end{defn}

\begin{defn}
Let $\tau$ and $\sigma$ be two topologies on a set $X$. We say that $\tau$ is \textbf{weaker} (or smaller, coarser) than $\sigma$ if $T \subseteq \sigma$. In this case, $\sigma$ is then said to be \textbf{stronger} (or larger, finer) than $\tau$.
\end{defn}

\begin{defn}
Let $X$ be any set. The collection $\tau = P\left (X\right )$ is a topology on $X$ and is called the \textbf{discrete topology} on $X$. Here $\left (X, \tau\right )$ is called the \textbf{discrete topological space}.
\end{defn}

\begin{defn}
Let $X$ be any set. The collection $\tau = \left\{ \varnothing, X \right\}$ is called the \textbf{indiscrete topology} on $X$. Here $\left (X, \tau\right )$ is called the \textbf{indiscrete topology}.
\end{defn}

\begin{defn}
    Let $X$ be any set. The collection $\tau = \left\{ A \subseteq X  :  X \setminus A \text{ is finite } \right\} \cup \left\{ \varnothing \right\}$ is called the \textbf{co-finite topology}.
\end{defn}

\begin{defn}
    Let $X$ be any set. The collection $\tau = \left\{ A \subseteq X  :  X \setminus A \text{ is countable } \right\} \cup \left\{ \varnothing \right\}$ is called the \textbf{co-countable topology}.
\end{defn}



\begin{defn}
A topology $\tau$ on a set $X$ is said to be \textbf{metrizable} if there exists a metric $d$ on $X$ such that the topology $\tau_d$ generated by the metric $d$ coincides with $\tau$.
\end{defn}

\begin{defn}
Two metrics defined on a set $X$ are said to be \textbf{equivalent} if they generate the same topology. 
In other words, $d_1$ and $d_2$ are equivalent if the collection of open sets in $\left (X, d_1\right )$ and $\left (X, d_2\right )$ are the same.
\end{defn}

\begin{defn}
The topology generated by the Euclidean metric on $\mathbb{R} ^n$ is called the \textbf{usual topology} on $\mathbb{R} ^n$.
For $Y \subseteq \mathbb{R} ^n$, the topology generated by the Euclidean metric is called the usual topology on $Y$.
\end{defn}

\begin{defn}
By a \textbf{neighbourhood} of a point $x$ in a topological space $\left (X, \tau\right )$, we mean an open set containing $x$.
\end{defn}

\begin{defn}
A subset $A$ of a topological space $\left (X, \tau\right )$ is said to be \textbf{closed} if $X \setminus A$ is open in $X$, that is $X \setminus A \in \tau$
\end{defn}

\begin{thm}
    \begin{itemize}
        \item[]
        \item  $\varnothing$ and $X$ are closed in $X$
        \item An intersection of any collection of closed sets is closed in $X$
        \item A union of a \textit{finite} collection of closed sets in $X$ is closed in $X$
    \end{itemize}
\end{thm}

\hhrule

\begin{defn}
    Let $X$ be a topological space. A collection $\beta$ of open subsets of $X$ is said to be a \textbf{basis} for the topology on $X$ if for every open set $U$ in $X$ and $x \in U$, there exists a $B \in \beta$ such that $x \in B \subseteq U$.

    Members of $\beta$ are called \textbf{basis open sets} corresponding to basis $\beta$.
\end{defn}

\begin{note}
    For any topological space $\left (X, \tau\right )$, $\tau$ is a basis for $\tau$.
\end{note}

\begin{note}
    In $\mathbb{R}$ the set $\left\{ \left (x - \varepsilon, x + \varepsilon\right )  :  x \in \mathbb{R}\text{ and }\varepsilon >0 \right\}$ is a basis for the usual topology.
\end{note}

\begin{note}
    For $\left (\mathbb{R}, \tau\right )$, where $\tau$ is the discrete topology, $\beta = \left\{ \left\{ x \right\}  :  x \in \mathbb{R} \right\}$ is a basis for $\tau$ on $\mathbb{R}$.
\end{note}

\begin{note}
    In $\mathbb{R}$ the set $\left\{ \left (x - \frac{1}{n}, x + \frac{1}{n}\right )  :  x \in \mathbb{R}\text{ and }n \in \mathbb{N} \right\}$ is a basis for the usual topology.
\end{note}

\begin{note}
    In $\mathbb{R}$ the set $\left\{ \left (a, b\right )  :  a,b \in \mathbb{Q}\text{ and }a < b \right\}$ is a basis for the usual topology. This is a countable basis for $\mathbb{R}$ with the usual topology.
\end{note}

\begin{note}
    Let $\left (X, \tau\right )$ be a metrizable space. Then $\beta = \left\{ B\left (x, \varepsilon\right )  :  x \in X, \varepsilon >0 \right\}$ is a basis for $\tau$.
\end{note}

\begin{defn}
    Let $\left (X, \tau\right )$ be a topological space. A collection $S \subseteq \tau$ is said to be a \textbf{subbasis} for the topology $\tau$ if for every open set $U$ in $\tau$ and $x \in U$, there exists a finite subcollection $\left\{ S_1, S_2, ..., S_n \right\}$ in $S$ such that $x \in \displaystyle\bigcap_{i = 1}^n S_i \subseteq U$

    Members of $S$ are called \textbf{subbasis open sets} corresponding to $S$.
\end{defn}

\begin{note}
    For a topological space $\left (X, \tau\right )$ a collection $S$ is a subbasis for the topology $\tau$ if and only if the collection of all finite intersections of members in $S$ forms a basis for the topology $\tau$.
\end{note}

\begin{thm}
    Let $\left (X, \tau\right )$ be a topological space and $\beta$ be a collection of open sets in $\left (X, \tau\right )$.

    Then $\beta$ is a basis for the topology $\tau$ on $X$ if and only if every open set $U$ in $\left (X, \tau\right )$ can be written as a union of members in $\beta$.
\end{thm}

\begin{thm}
    If $X$ is a set, a basis for a topology on $X$ is a collection $\beta$ of subsets of $X$ such that 
    \begin{itemize}
        \item for every $x \in X$ there exists $B \subset \beta$ containing $x$.
        \item if $x \in B_1 \cap B_2$, for some $B_1, B_2 \in \beta$, then there exists $B_3 \in \beta$ such that $x \in B_3 \subseteq B_1 \cap B_2$.
    \end{itemize}
\end{thm}

\begin{defn}
    The topology generated by the basis $\beta = \left\{ \left [a, b\right )  :  a, b \in \mathbb{R}, a < b \right\}$ is known as the \textbf{lower limit topology} on $\mathbb{R}$. 

    The space $\mathbb{R}$ equipped with the lower limit topology is known as the \textbf{Sorgenfrey line} $\mathbb{R}_l$.
\end{defn}

\begin{prop}
    The lower limit topology on $\mathbb{R}$ is strictly finer than the usual topology on $\mathbb{R}$
\end{prop}

\begin{defn}
    Let $K = \left\{ \frac{1}{n}  :  n \in \mathbb{N} \right\}$. Consider the basis \\ $\beta = \left\{ \left (a, b\right ) \setminus K  :  a,b \in \mathbb{Q}, a < b \right\} \cup \left\{ \left (a,b\right )  :  a,b \in \mathbb{R}, a < b \right\}$.  The topology generated by the basis $\beta$ is known as the \textbf{$k$-topology} on $\mathbb{R}$.
\end{defn}

\begin{prop}
    The $k$-topology on $\mathbb{R}$ is strictly finer than the usual topology on $\mathbb{R}$.
\end{prop}

\begin{prop}
    The lower limit topology and the $k$-topology on $\mathbb{R}$ are incomparable.
\end{prop}


\begin{defn}
    Let $X$ be a set. Let $S$ be a collection of subsets of $X$ whose union equals $X$. Any such collection is called a \textbf{subbasis} for a topology $X$.
\end{defn}

\begin{thm}
    Let $S$ be a subbasis for a topology on $X$. Then
    \\ $\tau = \{ U \subseteq X  :  $ for every $ x \in U $ there exists a finite number of members $ S_1, S_2, ..., S_n \in S $ such that $ x \in S_1 \cap S_2 \cap ... \cap S_n \subseteq U \}$ is the topology generated by the subbasis $S$.
\end{thm}

\begin{thm}
    Let $\beta$ be a basis for a topology on $X$. Then the topology generated by $\beta$ is equal to the intersection of all topologies on $X$ that contains $\beta$.
\end{thm}

\begin{thm}
    If $\beta$ is a basis for a topology on a set $X$, then the topology generated by $\beta$ on $\tau$ is the smallest topology on $X$ that contains $\beta$.
\end{thm}

\hhrule

\begin{defn}
    Let $X$ be a set. A function $f \colon \mathbb{N} \to X$ is called a \textbf{sequence} in $X$.
\end{defn}

\begin{defn}
    Let $\left (X, Y\right )$ be a topological space. A sequence $\left (X_n\right )$ is said to \textbf{converge} to $x \in X$ if for every neighbourhood $U$ of $x$, there exists $n_0 \in \mathbb{N}$ such that $x_n \in U \;\forall n \geq n_0$
\end{defn}

\begin{defn}
    A sequence $\left (x_n\right )$ is said to be \textbf{eventually constant} if there exists $n_0 \in \mathbb{N}$ such that $x_n = x_{n_0} \;\forall n \geq n_0$.
\end{defn}

\begin{thm}
    In any topological space, every eventually constant sequence is convergent.
\end{thm}

\begin{defn}
    Let $f \colon \mathbb{N} \to X$ be a sequence in $X$. Then for any strictly increasing function $g \colon \mathbb{N} \to \mathbb{N}$, the composition $f \circ g \colon \mathbb{N} \to X$ is called a \textbf{subsequence} of $f$.
\end{defn}

\begin{thm}
    In a topological space $\left (X, \tau\right )$, every subsequence of a convergent sequence is convergent.
\end{thm}

\begin{defn}
    Let $\mathcal{A}$ be a non-empty set. A relation $\leq$ on a set $A$ is called a \textbf{partial order relation} if the following condition holds for all $\alpha, \beta, \gamma$ in $A$:
    \begin{enumerate}
        \item reflexive: $\alpha \leq \alpha$
        \item anti-symmetric: $\alpha \leq \beta$ and $\beta \leq \alpha$ $\implies$ $\alpha = \beta$
        \item transitive: $\alpha \leq \beta$ and $\beta \leq \gamma$ $\implies$ $\alpha \leq \gamma$
    \end{enumerate}
\end{defn}

\begin{defn}
    A \textbf{directed set} $J$ is a set with a partial order relation $\leq$ such that for each pair $\alpha$ and $\beta$ of $J$, there exists a $Y \in J$ such that $\alpha \leq Y$ and $\beta \leq Y$.
\end{defn}

\begin{defn}
    A \textbf{net} in $X$ is a function $f$ from a directed set $J$ to $X$.
\end{defn}

\begin{note}
Every sequence is a net.
\end{note}

\begin{defn}
    Let $J$ be a directed set with a partial order relation '$\leq$'. A subset $K$ of $J$ is said to be \textbf{cofinal} in $J$ if for each $\alpha \in J$, there exists $\beta \in K$ such that $\alpha \leq \beta$.
\end{defn}

\begin{prop}
    If $g \colon \mathbb{N} \to \mathbb{N}$ is a strictly increasing function then $g\left (\mathbb{N}\right )$ is cofinal in $\mathbb{N}$.
\end{prop}

\begin{thm}
    If $J$ is a directed set and $K$ is cofinal in $J$, then $K$ is a directed set.
\end{thm}

\begin{defn}
    Let $f \colon J \to X$ be a net in $X$. If $I$ is a directed set and $g \colon I \to J$ such that
    \begin{itemize}
        \item $i < j \implies g\left (i\right ) < g\left (j\right ) \;\forall i, j \in I$
        \item $g\left (I\right )$ is cofinal in $J$
    \end{itemize}
    Then $f \circ g \colon I \to X$ is called a \textbf{subnet} of $f$.
\end{defn}

\begin{defn}
    The net $\left (X_\alpha\right )_{\alpha \in J}$ is said to \textbf{converge} to a point $x \in X$ if for every neighbourhood $U$ of $x$, there exists $\alpha_0 \in J$ such that $x_\alpha \in U$ for all $\alpha \geq \alpha_0$.
\end{defn}

\hhrule 

\begin{defn}
    Let $\left (X, \tau\right )$ be a topological space and $S \subseteq X$. A point $x \in X$ is called a \textbf{closure point} of $S$ if for every neighbourhood $U$ of $x$, we have $U \cap S \neq  \varnothing$.

    The set of all closure points of $S$ is called the \textbf{closure} of $A$ and is denoted $\text{cl}\left (A\right )$ or $\bar{A}$.
\end{defn}

\begin{defn}
    A point $x \in X$ is said to be a \textbf{limit point} of $S$ if for every neighbourhood of $x$, we have $\left (U \cap S\right ) \setminus \left\{ x \right\} \neq  \varnothing $.

    The set of all limit points is called the \textbf{derived set} and is denoted $S'$
\end{defn}

\begin{thm}
    Let $X$ be a topological space and $S \subseteq X$. Then $\bar{S}$ is the smallest closed set in $X$ that contains $S$.
\end{thm}

\begin{prop}
    A subset of a topological space $X$ is closed if and only if $S = \bar{S}$
\end{prop}

\begin{thm}
    Let $\left (X, \tau\right )$ be a metrizable space and $S \subseteq X$. Then $x \in \bar{S}$ if and only if there exists a sequence $\left\langle x_n\right\rangle $ in $S$ such that $x_n \to x$.
\end{thm}

\begin{thm}
    Let $X$ be a topological space and $S \subseteq X$. Then show that $x \in \bar{S}$ if and only if there exists a net $\left\langle  x_\lambda \right\rangle $ in $S$ such that $ x_\lambda \to x$
\end{thm}

\begin{defn}
        Let $X$ be a topological space and $A \subseteq X$. A point $x \in X$ is said to be an \textbf{interior point} of $A$ if there exists a neighbourhood $U$ of $X$ such that $U \subseteq A$ (or) if there exists an open set $U$ in $X$ such that $x \in U \subseteq A$.

        The set of all interior points of $A$ is called the \textbf{interior} of $A$ and is denoted $A ^\mathrm{o}$.
\end{defn}

\begin{thm}
    Let $X$ be a topological space and $S \subseteq X$. Then $S ^\mathrm{o}$ is the largest open set contained in $S$.
\end{thm}

\begin{thm}
    Let $X$ be a topological space and $S \subseteq X$. Then $S$ is open if and only if $S = S ^\mathrm{o}$.
\end{thm}

\hhrule 

\begin{defn}
    Let $\left (X, \tau_1\right )$ and $\left (Y, \tau_2\right )$ be two topological spaces. The collection $\mathcal{B} = \tau_1 \times \tau_2 = \left\{ u \times v  :  u \in \tau_1, v \in \tau_2 \right\}$ is a basis for a topology on $X \times Y$.
    The topology generated by $\mathcal{B}$ is called the \textbf{product topology} on $X \times Y$.
\end{defn}

\begin{thm}
    Let $\left (X, \tau\right )$ and $\left (Y, \sigma\right )$ be two topological spaces. Let $\mathcal{B}$ be a basis for $\tau$ and $\mathcal{C}$ be a basis for $\sigma$.
    Then the collection $\mathcal{D} = \left\{ U \times V  :  U \in \mathcal{B}, V \in \mathcal{C} \right\}$ is a basis for the product topology on $X \times Y$.
\end{thm}

\begin{result}
    The product topology on $\mathbb{R} \times \mathbb{R}$ coincides with the usual topology on $\mathbb{R}^2$
\end{result}

\begin{note}
    Missed quotient topology, check.
\end{note}
\hhrule

\begin{defn}
    Let $(X, \tau)$ be a topological space and $Y \subseteq X$.
    
    The collection $\tau_Y$ is a topology on $Y$ called the \textbf{subspace topology} and with this topology, $Y$ is called a \textbf{subspace} of $X$.
\end{defn}

\begin{thm}
    Let $Y$ be a subspace of $X$. Then a subset $U$ of $Y$ is open in $Y$ iff $U = V \cap Y$ for some open set $V$ in $X$.
\end{thm}

\begin{thm}
    If $\mathscr{B}$ is a basis for the topology on $X$, then the collection $\mathscr{B}_Y = \{ B \cap Y  | B \in \mathscr{B} \}$ is a basis for the subspace topology on $Y$.
\end{thm}



\begin{thm}
    Let $Y$ be an open subspace of $X$. Then a subset $U$ of $Y$ is open in $Y$ iff $U$ is open in $X$.
\end{thm}

\begin{thm}
    Let $Y$ be a closed subspace of a topological space $X$. Then a subset $C$ of $Y$ is closed in $Y$ iff it is closed in $X$.
\end{thm}

\begin{thm}
    Let $X$ be a topological space and $Y$ be a subspace of $X$.

    Let $C \subseteq Y$. Then $C$ is closed in $Y$ iff there is a closed set $D$ in $X$ such that $Y \cap D = C$
\end{thm}

\begin{thm}
    Let $X$ be a topological space and $Y$ be a closed subspace of $X$. Let $C \subseteq Y$. Then $C$ is closed in $Y$ iff $C$ is closed in $X$.
\end{thm}

\begin{thm}
    If $A$ is a subspace of $X$ and $B$ is a subspace of $Y$, then the product topology on $A \times B$ is the same as the topology $A \times B$ inherits as a subspace of $X \times Y$.
\end{thm}

\begin{thm}
    Let $X$ be a topological space and $Y$ be a subspace of $X$. Let $A \subseteq Y$. Then $\text{cl}_X(A) \cap Y = \text{cl}_Y(A)$
\end{thm}

\hhrule

\begin{defn}
    A function $f:(X, \tau) \to (Y, \sigma)$ is said to be \textbf{continuous} at $x_0 \in X$ if for every neighbourhood $U$ of $f(x_0)$, there exists a neighbourhood $V$ of $x_0$ such that $f(V) \subseteq U$.

    The function $f$ is said to be \textbf{continuous} if it is continuous at each point of $X$.
\end{defn}

\begin{thm}
    Let $f:(X,\tau) \to (Y, \sigma)$ be a function from a topological space $(X,\tau)$ to a topological space $(Y,\sigma)$. Then the following are equivalent:
    \begin{itemize}
        \item $f$ is continuous on $X$
        \item $f^{-1}(U)$ is an open subset of $X$ whenever $U$ is open in $X$
        \item $f(\overline{A}) = \overline{f(A)}$ for every $A \subseteq X$
        \item $f^{-1}(C)$ is a closed subset of $X$ whenever $C$ is closed in $Y$
        \item The net $x_\lambda \to x$ in $X$ $\implies f(x_\lambda) \to f(x)$ in $Y$
    \end{itemize}
\end{thm}

\begin{thm}
    The composition of two continuous functions is continuous.
\end{thm}

\begin{thm}
    The restriction of a continuous function on a topological space to a subspace is continuous.
\end{thm}

\begin{thm}
    Let $(X, \tau)$ and $(Y, \sigma)$ be topological spaces and $Z = X \times Y$. 
    For a net $(z_\lambda) = (x_\lambda, y_\lambda)$ in $Z$, $z_\lambda \to z \in Z \iff x_\lambda \to x \in X \text{ and } y_\lambda \to y \in Y$.
\end{thm}

\begin{defn}
    Let $X$ and $Y$ be topological spaces and let $f: X \to Y$ be a bijection. If both the function $f$ and the inverse function $f^{-1}: Y \to X$ are continuous, then $f$ is called a \textbf{homeomorphism}.
\end{defn}

\hhrule


\begin{defn}
    Let $\{ X_\alpha \}_{\alpha \in I}$ be an indexed family of topological spaces. For the product space $X =  \prod_{\alpha \in I} X_\alpha$, we take the basis $\mathscr{B} = \{ \prod_{\alpha \in I} U_\alpha | U_\alpha \text{ is open in } X_\alpha \}$. The topology generated by this basis is called the \textbf{box topology}. 
\end{defn}

\begin{note}
    The above is a direct generalization of the earlier product topology defined on product of two spaces.
\end{note}

\begin{thm}
    A subset $W$ of $X$ is open in $X$ with the box topology iff for every $x = (x_\alpha)_{\alpha \in I}$, there exists an open set $U_\alpha$ in $X_\alpha$ such that $(x_\alpha)_{\alpha \in I} \in \prod_{\alpha \in I} U_\alpha \subseteq W$
\end{thm}

\begin{defn}
Let $\{ X_\alpha \}_{\alpha \in I}$ be an indexed family of topological spaces. For the product space $X =  \prod_{\alpha \in I} X_\alpha$, we take the basis $\mathscr{C} = \{ \prod_{\alpha \in I} U_\alpha | U_\alpha \text{ is open in } X, U_\alpha = X_\alpha \text{ for all but finitely many } \alpha \in I \}$. 

    The topology generated by this basis is called the \textbf{product topology} on $X$.
\end{defn}

\begin{note}
    Unless mentioned otherwise, we usually consider the product topology on the product space $X$.
\end{note}

\begin{thm}
    The product topology on $X$ is weaker than the box topology on $X$.
\end{thm}

\begin{note}
    When the index $I$ is finite, the product topology and the box topology coincide.
\end{note}

\begin{defn}
    Let $\{ X_\alpha \}_{\alpha \in I}$ be an indexed family of topological spaces and $X = \prod_{\alpha \in I} X_\alpha$ be the product space. 

    For every $\beta \in I$, the \textbf{$\beta$-th projection map} $\Pi_\beta: X \to X_\beta$ is defined as $\Pi_\beta ( (x_\alpha)_{\alpha \in I} ) = x_\beta$
\end{defn}

\begin{thm}
Let $\{ X_\alpha \}_{\alpha \in I}$ be an indexed family of topological spaces and $X = \prod_{\alpha \in I} X_\alpha$ be the product space. 

    The collection $\mathscr{S} = \{ \Pi^{-1}_\beta (U_\beta) | U_\beta \text{ is open in } X_\beta, \beta \in I \}$is a subbasis for the product topology on $X$.
\end{thm}

\begin{thm}
Let $\{ X_\alpha \}_{\alpha \in I}$ be an indexed family of topological spaces and $X = \prod_{\alpha \in I} X_\alpha$ be the product space equipped with the product topology.

    Then a function $f: A \to X$ is continuous iff $f_\alpha = \Pi_\alpha \circ f : A \to X_\alpha$ is continuous for each $\alpha \in I$.
\end{thm}

\begin{thm}
    The topologies on $\mathbb{R}^n$ induced by the euclidean metric and the square norm are the same as the product topology on $\mathbb{R}^n$.
\end{thm}

\begin{defn}
    Consider the standard bounded metric on $\mathbb{R}$ defined as $\overline{d} = \min\{ |a-b|, 1\}$.
    Given an index set $I$ and points $x = (x_\alpha)_{\alpha \in I}$ and $y = (y_\alpha)_{\alpha \in I}$, the metric $\rho$ on $\mathbb{R}^I$ defined  as $\rho(x,y) = \sup \{ \overline{d}(x_\alpha, y_\alpha) | \alpha \in I \}$ is known as the \textbf{uniform metric} on $\mathbb{R}^I$.

    The topology generated by the uniform metric is called the \textbf{uniform topology} on $\mathbb{R}^I$.
\end{defn}

\begin{thm}
    The uniform topology on $\mathbb{R}^I$ is weaker than the box topology and stronger than the product topology. 
    All these topologies are different if $I$ is infinite.
\end{thm}

\begin{defn}
    Let $X$ and $Y$ be topological spaces and let $p: X \to Y$ be a surjective map. 

    The map $p$ is said to be a \textbf{quotient map} provided a subset $U$ of $Y$ is open in $Y$ iff $p^{-1}(U)$ is open in $X$.
\end{defn}

\begin{note}
    We can replace 'open' in the above definition with 'closed'.
\end{note}

\begin{note}
    Every quotient map is a continuous map.
\end{note}

\begin{defn}
    A map $f: X \to Y$ is said to be \textbf{open} if for each open set $U$ in $X$, $f(U)$ is open in $Y$.
    A map $f: X \to Y$ is said to be \textbf{closed} if for each closed set $C$ in $X$, $f(C)$ is closed in $Y$.
\end{defn}


\begin{note}
    A surjective continuous map which is either open or closed is a quotient map. Note, there are quotient maps which are neither open nor closed.
\end{note}

\begin{defn}
    If $X$ is a topological space and $A$ is any set, and if $p: X \to A$ is a surjective map, then there exists exactly one topology $\tau$ on $A$ relative to which $p$ is a quotient map. This topology is known as the \textbf{quotient topology} induced by $p$.
\end{defn}

\begin{defn}
    Let $X$ be a topological space and $\sim$ be an equivalence relation on $X$. 

    Define a map $p: X \to \frac{X}{\sim}$ as $p(x) = [x]$ where $[x]$ is the equivalence class of $x$ under $\sim$.

    Let $\tau$ be the quotient topology induced by $p$ on $\frac{X}{\sim}$. Then the space $(\frac{X}{\sim}, \tau)$ is called the \textbf{quotient space} of $X$.
\end{defn}


\hhrule



\end{document}

