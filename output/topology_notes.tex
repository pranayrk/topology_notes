\documentclass[12pt,twoside]{report}
\usepackage[utf8]{inputenc}
\usepackage{amsmath,amsfonts,graphicx,amsthm}
\usepackage{epsfig,amssymb}
\usepackage{caption}
\usepackage{fancyhdr}
\usepackage{lipsum}
\usepackage{thmtools}
\usepackage{enumitem}
\usepackage[top=1.25in, bottom=1.25in,left=1.5in, right=1.25in]{geometry}

\newtheorem{defn}{Definition}
\newtheorem{note}{Note}
\newtheorem{notation}[subsection]{Notation}
\newtheorem{thm}{Theorem}
\newtheorem{corollary}[subsection]{Corollary}
\newtheorem{eg}[subsection]{Example}
\newtheorem{lemma}[subsection]{Lemma}
\newtheorem{result}{Result}
\declaretheoremstyle[bodyfont=\normalfont]{normalbody}

\declaretheorem[style=normalbody,name=]{ex}
\newtheorem{remark}[subsection]{Remark}
\newtheorem{prop}[subsection]{Proposition}
\newenvironment*{ans}{\textbf{ans.}\space\em\\}{\par}
\newenvironment*{source}{\hfill\scriptsize\textbf{Source.}\space}{\par}

\declaretheoremstyle[headfont=\normalfont]{normalhead}

\makeatletter
\newcommand*{\rom}[1]{\expandafter\@slowromancap\romannumeral #1@}
\makeatother

\widowpenalties 1 10000
\raggedbottom
\setlength{\parindent}{0pt}
\renewcommand{\chaptername}{}

\setlength{\oddsidemargin}{36pt}
\setlength{\evensidemargin}{36pt}

\begin{document}

\tableofcontents
\newpage
\pagenumbering{arabic}

\chapter{Definitions and Theory}
\line(1,0){360}

\section{Topological Spaces}

\begin{defn}
A \textbf{topology} on a set $X$ is a collection $\tau$ of subsets of $X$ satisfying:
    \begin{enumerate}
        \item $\varnothing, X \in  \tau$
        \item  An intersection of finite subcollections of $\tau$ is in $\tau$
        \item A union of any subcollection of $\tau$ is in $\tau$
    \end{enumerate}
The ordered pair $\left (X, \tau\right )$ is called a \textbf{topological space}.
\end{defn}


\begin{defn}
    Let $\left (X, \tau\right )$ be a topological space. An \textbf{open subset} of $X$ is a member of $\tau$.
\end{defn}

\begin{defn}
Let $\tau$ and $\sigma$ be two topologies on a set $X$. We say that $\tau$ is \textbf{weaker} (or smaller, coarser) than $\sigma$ if $T \subseteq \sigma$. In this case, $\sigma$ is then said to be \textbf{stronger} (or larger, finer) than $\tau$.
\end{defn}

\begin{defn}
Let $X$ be any set. The collection $\tau = P\left (X\right )$ is a topology on $X$ and is called the \textbf{discrete topology} on $X$. Here $\left (X, \tau\right )$ is called the \textbf{discrete topological space}.
\end{defn}

\begin{defn}
Let $X$ be any set. The collection $\tau = \left\{ \varnothing, X \right\}$ is called the \textbf{indiscrete topology} on $X$. Here $\left (X, \tau\right )$ is called the \textbf{indiscrete topology}.
\end{defn}

\begin{defn}
    Let $X$ be any set. The collection $\\tau = \left\{ A \subseteq X  :  X \setminus A \text{ is finite } \right\} \cup \left\{ \varnothing \right\}$ is called the \textbf{co-finite topology}.
\end{defn}

\begin{defn}
    Let $X$ be any set. The collection $\\tau = \left\{ A \subseteq X  :  X \setminus A \text{ is countable } \right\} \cup \left\{ \varnothing \right\}$ is called the \textbf{co-finite topology}.
\end{defn}



\begin{defn}
A topology $\tau$ on a set $X$ is said to be \textbf{metrizable} if there exists a metric $d$ on $X$ such that the topology $\tau_d$ generated by the metric $d$ coincides with $\tau$.
\end{defn}

\begin{defn}
Two metrics defined on a set $X$ are said to be \textbf{equivalent} if they generate the same topology. 
In other words, $d_1$ and $d_2$ are equivalent if the collection of open sets in $\left (X, d_1\right )$ and $\left (X, d_2\right )$ are the same.
\end{defn}

\begin{defn}
The topology generated by the Euclidean metric on $\mathbb{R} ^n$ is called the \textbf{usual topology} on $\mathbb{R} ^n$.
For $Y \subseteq \mathbb{R} ^n$, the topology generated by the Euclidean metric is called the usual topology on $Y$.
\end{defn}

\begin{defn}
By a \textbf{neighbourhood} of a point $x$ in a topological space $\left (X, \tau\right )$, we mean an open set containing $x$.
\end{defn}

\begin{defn}
A subset $A$ of a topological space $\left (X, \tau\right )$ is said to be \textbf{closed} if $X \setminus A$ is open in $X$, that is $X \setminus A \in \tau$
\end{defn}

\begin{thm}
    \begin{itemize}
        \item[]
        \item  $\varnothing$ and $X$ are closed in $X$
        \item An intersection of any collection of closed sets is closed in $X$
        \item A union of a \textit{finite} collection of closed sets in $X$ is closed in $X$
    \end{itemize}
\end{thm}

\begin{defn}
    Let $X$ be a topological space and $A \subseteq X$. A point $x \in X$ is said to be an \textbf{interior point} of $A$ if there exists a neighbourhood $U$ of $X$ such that $U \subseteq A$ (or) if there exists an open set $U$ in $X$ such that $x \in U \subseteq A$. 

The set of all interior points of $A$ is called the \textbf{interior} of $A$ and is denoted $A ^\mathrm{o}$.
\end{defn}

\begin{thm}
    Let $\left (X, d\right )$ be a topological space and $S \subseteq X$. Then $S ^\mathrm{o}$ is the largest open set in $X$ that is contained in $S$.
\end{thm}

\begin{thm}
    Let $X$ be a topological space and $S \subseteq X$ Then $S$ is open if and only if $S = S ^\mathrm{o}$.
\end{thm}

\section{Bases and Subbases}

\begin{defn}
    Let $X$ be a topological space. A collection $\beta$ of open subsets of $X$ is said to be a \textbf{basis} for the topology on $X$ if for every open set $U$ in $X$ and $x \in U$, there exists a $B \in \beta$ such that $x \in B \subseteq U$.

    Members of $\beta$ are called \textbf{basis open sets} corresponding to basis $\beta$.
\end{defn}

\begin{note}
    For any topological space $\left (X, \tau\right )$, $\tau$ is a basis for $\tau$.
\end{note}

\begin{note}
    In $\mathbb{R}$ the set $\left\{ \left (x - \varepsilon, x + \varepsilon\right )  :  x \in \mathbb{R}\text{ and }\varepsilon >0 \right\}$ is a basis for the usual topology.
\end{note}

\begin{note}
    For $\left (\mathbb{R}, \tau\right )$, where $\tau$ is the discrete topology, $\beta = \left\{ \left\{ x \right\}  :  x \in \mathbb{R} \right\}$ is a basis for $\tau$ on $\mathbb{R}$.
\end{note}

\begin{note}
    In $\mathbb{R}$ the set $\left\{ \left (x - \frac{1}{n}, x + \frac{1}{n}\right )  :  x \in \mathbb{R}\text{ and }n \in \mathbb{N} \right\}$ is a basis for the usual topology.
\end{note}

\begin{note}
    In $\mathbb{R}$ the set $\left\{ \left (a, b\right )  :  a,b \in \mathbb{Q}\text{ and }a < b \right\}$ is a basis for the usual topology. This is a countable basis for $\mathbb{R}$ with the usual topology.
\end{note}

\begin{note}
    Let $\left (X, \tau\right )$ be a metrizable space. Then $\beta = \left\{ B\left (x, \varepsilon\right )  :  x \in X, \varepsilon >0 \right\}$ is a basis for $\tau$.
\end{note}

\begin{defn}
    Let $\left (X, \tau\right )$ be a topological space. A collection $S \subseteq \tau$ is said to be a \textbf{subbasis} for the topology $\tau$ if for every open set $U$ in $\tau$ and $x \in U$, there exists a finite subcollection $\left\{ S_1, S_2, ..., S_n \right\}$ in $S$ such that $x \in \displaystyle\bigcap_{i = 1}^n S_i \subseteq U$

    Members of $S$ are called \textbf{subbasis open sets} corresponding to $S$.
\end{defn}

\begin{note}
    For a topological space $\left (X, \tau\right )$ a collection $S$ is a subbasis for the topology $\tau$ if and only if the collection of all finite intersections of members in $S$ forms a basis for the topology $\tau$.
\end{note}

\begin{thm}
    Let $\left (X, \tau\right )$ be a topological space and $\beta$ be a collection of open sets in $\left (X, \tau\right )$.

    Then $\beta$ is a basis for the topology $\tau$ on $X$ if and only if every open set $U$ in $\left (X, \tau\right )$ can be written as a union of members in $\beta$.
\end{thm}

\begin{thm}
    If $X$ is a set, a basis for a topology on $X$ is a collection $\beta$ of subsets of $X$ such that 
    \begin{itemize}
        \item for every $x \in X$ there exists $B \subset \beta$ containing $x$.
        \item if $x \in B_1 \cap B_2$, for some $B_1, B_2 \in \beta$, then there exists $B_3 \in \beta$ such that $x \in B_3 \subseteq B_1 \cap B_2$.
    \end{itemize}
\end{thm}

\begin{defn}
    The topology generated by the basis $\beta = \left\{ \left [a, b\right )  :  a, b \in \mathbb{R}, a < b \right\}$ is known as the \textbf{lower limit topology} on $\mathbb{R}$. 

    The space $\mathbb{R}$ equipped with the lower limit topology is known as the \textbf{Sorgenfrey line} $\mathbb{R}_l$.
\end{defn}

\begin{prop}
    The lower limit topology on $\mathbb{R}$ is strictly finer than the usual topology on $\mathbb{R}$
\end{prop}

\begin{defn}
    Let $K = \left\{ \frac{1}{n}  :  n \in \mathbb{N} \right\}$. Consider the basis \\ $\beta = \left\{ \left (a, b\right ) \setminus K  :  a,b \in \mathbb{Q}, a < b \right\} \cup \left\{ \left (a,b\right )  :  a,b \in \mathbb{R}, a < b \right\}$.  The topology generated by the basis $\beta$ is known as the \textbf{$k$-topology} on $\mathbb{R}$.
\end{defn}

\begin{prop}
    The $k$-topology on $\mathbb{R}$ is strictly finer than the usual topology on $\mathbb{R}$.
\end{prop}

\begin{prop}
    The lower limit topology and the $k$-topology on $\mathbb{R}$ are incomparable.
\end{prop}


\begin{defn}
    Let $X$ be a set. Let $S$ be a collection of subsets of $X$ whose union is in $X$. Any such collection is called a \textbf{subbasis} for a topology $X$. (TODO: Verify)
\end{defn}

\begin{thm}
    Let $S$ be a subbasis for a topology on $X$. Then
    \\ $\tau = \{ U \subseteq X  :  $ for every $ x \in U $ there exists a finite number of members $ S_1, S_2, ..., S_n \in S $ such that $ x \in S_1 \cap S_2 \cap ... \cap S_n \subseteq U \}$ is the topology generated by the subbasis $S$.
\end{thm}

\begin{thm}
    Let $\beta$ be a basis for a topology on $X$. Then the topology generated by $\beta$ is equal to the intersection of all topologies on $X$ that contains $\beta$.
\end{thm}

\begin{thm}
    If $\beta$ is a basis for a topology on a set $X$, then the topology generated by $\beta$ on $\tau$ is the smallest topology on $X$ that contains $\beta$.
\end{thm}

\begin{defn}
    Let $X$ be a set. A function $f \colon \mathbb{N} \to X$ is called a \textbf{sequence} in $X$.
\end{defn}

\begin{defn}
    Let $\left (X, Y\right )$ be a topological space. A sequence $\left (X_n\right )$ is said to \textbf{converge} to $x \in X$ if for every neighbourhood $U$ of $x$, there exists $n_0 \in \mathbb{N}$ such that $x_n \in U \;\forall n \geq n_0$
\end{defn}

\begin{defn}
    A sequence $\left (x_n\right )$ is said to be \textbf{eventually constant} if there exists $n_0 \in \mathbb{N}$ such that $x_n = x_{n_0} \;\forall n \geq n_0$.
\end{defn}

\begin{thm}
    In any topological space, every eventually constant sequence is convergent.
\end{thm}

\begin{defn}
    Let $f \colon \mathbb{N} \to X$ be a sequence in $X$. Then for any strictly increasing function $g \colon \mathbb{N} \to \mathbb{N}$, the composition $f \circ g \colon \mathbb{N} \to X$ is called a \textbf{subsequence} of $f$.
\end{defn}

\begin{thm}
    In a topological space $\left (X, \tau\right )$, every subsequence of a convergent sequence is convergent.
\end{thm}

\begin{defn}
    Let $\mathcal{A}$ be a non-empty set. A relation $\leq$ on a set $A$ is called a \textbf{partial order relation} if the following condition holds for all $\alpha, \beta, \gamma$ in $A$:
    \begin{enumerate}
        \item reflexive: $\alpha \leq \alpha$
        \item anti-symmetric: $\alpha \leq \beta$ and $\beta \leq \alpha$ $\implies$ $\alpha = \beta$
        \item transitive: $\alpha \leq \beta$ and $\beta \leq \gamma$ $\implies$ $\alpha \leq \gamma$
    \end{enumerate}
\end{defn}

\begin{defn}
    A \textbf{directed set} $J$ is a set with a partial order relation $\leq$ such that for each pair $\alpha$ and $\beta$ of $J$, there exists a $Y \in J$ such that $\alpha \leq Y$ and $\beta \leq Y$.
\end{defn}

\begin{defn}
    A \textbf{net} in $X$ is a function $f$ from a directed set $J$ to $X$.
\end{defn}

Note that every sequence is a net.

\begin{defn}
    Let $J$ be a directed set with a partial order relation '$\leq$'. A subset $K$ of $J$ is said to be \textbf{cofinal} in $J$ if for each $\alpha \in J$, there exists $\beta \in K$ such that $\alpha \leq \beta$.
\end{defn}

\begin{prop}
    If $g \colon \mathbb{N} \to \mathbb{N}$ is a strictly increasing function then $g\left (\mathbb{N}\right )$ is cofinal in $\mathbb{N}$.
\end{prop}

\begin{thm}
    If $J$ is a directed set and $K$ is cofinal in $J$, then $K$ is a directed set.
\end{thm}

\begin{defn}
    Let $f \colon J \to X$ be a net in $X$. If $I$ is a directed set and $g \colon I \to J$ such that 
    \begin{itemize}
        \item $i < j \implies g\left (i\right ) < g\left (j\right ) \;\forall i, j \in I$
        \item $g\left (I\right )$ is cofinal in $J$
    \end{itemize}
    Then $f \circ g \colon I \to X$ is called a \textbf{subnet} of $f$.
\end{defn}

\begin{defn}
    The net $\left (X_\alpha\right )_{\alpha \in J}$ is said to \textbf{converge} to a point $x \in X$ if for every neighbourhood $U$ of $x$, there exists $\alpha_0 \in J$ such that $x_\alpha \in U$ for all $\alpha \geq \alpha_0$.
\end{defn}

\begin{defn}
    Let $\left (X, \tau\right )$ be a topological space and $S \subseteq X$. A point $x \in X$ is called a \textbf{closure point} of $S$ if for every neighbourhood $U$ of $x$, we have $U \cap S \neq  \varnothing$.

    THe set of all closure points of $S$ is called the \textbf{closure} of $A$ and is denoted $\text{cl}\left (A\right )$ or $\bar{A}$.
\end{defn}

\begin{defn}
    A point $x \in X$ is said to be a \textbf{limit point} of $S$ if for every neighbourhood of $x$, we have $\left (U \cap S\right ) \setminus \left\{ x \right\} \neq  \varnothing $.

    The set of all limit points is called the \textbf{derived set} and is denoted $S'$
\end{defn}

\begin{thm}
    Let $X$ be a topological space and $S \subseteq X$. Then $\bar{S}$ is the smallest closed set in $X$ that contains $S$.
\end{thm}

\begin{prop}
    A subset of a topological space $X$ is closed if and only if $S = \bar{S}$
\end{prop}

\begin{thm}
    Let $\left (X, \tau\right )$ be a metrizable space and $S \subseteq X$. Then $x \in \bar{S}$ if and only if there exists a sequence $\left\langle x_n\right\rangle $ in $S$ such that $x_n \to x$.
\end{thm}

\begin{thm}
    Let $X$ be a topological space and $S \subseteq X$. Then show that $x \in \bar{S}$ if and only if there exists a net $\left\langle  x_\lambda \right\rangle $ in $S$ such that $ x_\lambda \to x$
\end{thm}

\begin{thm}
    Let $X$ be a topological space and $S \subseteq X$. Then $S ^\mathrm{o}$ is the largest open set contained in $S$.
\end{thm}

\begin{thm}
    Let $X$ be a topological space and $S \subseteq X$. Then $S$ is open if and only if $S = S ^\mathrm{o}$.
\end{thm}

\begin{defn}
    Let $\left (X, \tau_1\right )$ and $\left (Y, \tau_2\right )$ be two topological spaces. The collection $\mathcal{B} = \tau_1 \times \tau_2 = \left\{ u \times v  :  u \in \tau_1, v \in \tau_2 \right\}$ is a basis for a topology on $X \times Y$.
    The topology generated by $\mathcal{B}$ is called the product topology on $X \times Y$.
\end{defn}

\begin{thm}
    Let $\left (X, \tau\right )$ and $\left (Y, \sigma\right )$ be two topological spaces. Let $\mathcal{B}$ be a basis for $\tau$ and $\mathcal{C}$ be a basis for $\sigma$.
    Then the collection $\mathcal{D} = \left\{ U \times V  :  U \in \mathcal{B}, V \in \mathcal{C} \right\}$ is a basis for the product topology on $X \times Y$.
\end{thm}

\chapter{Exercises}
\line(1,0){360}

\begin{samepage}
\begin{ex}
Let $\left (X, d\right )$ be a metric space and $\tau_d$ be the collection of all open subsets of $X$. 
Show that $\tau_d$ is a topology on $X$
\vspace{0.5cm}
\textit{This is called the topology on $X$ generated by $d$}
\end{ex}
\begin{source}
Class, Aug 07
\end{source}
\end{samepage}

\begin{samepage}
\begin{ex}
Let $X$ be any set and $A \subseteq X$. Then show that 
$\tau = \left\{ \varnothing, A , X \right\}$ is a topology on $X$.
\end{ex}
\begin{source}
Class, Aug 07
\end{source}
\end{samepage}

\begin{samepage}
\begin{ex}
Let $X$ be any set. Let 
$\tau = \left\{ A \subseteq X  :  X \setminus A \text{ is finite } \right\} \cup \left\{ \varnothing \right\} $
Then show that $\tau$ is a topology on $X$.
\end{ex}
\begin{source}
Class, Aug 07
\end{source}
\end{samepage}

\begin{samepage}
\begin{ex}
Let $X$ be any set and 
$\tau = \left\{ A \subseteq X  :  X \setminus A \text{ is countable } \right\} \cup \left\{ \varnothing \right\} $.
Then show that $\tau$ is a topology on $X$.
\end{ex}
\begin{source}
Class, Aug 07
\end{source}
\end{samepage}

\begin{samepage}
\begin{ex}
Show that every discrete topological space is metrizable
\end{ex}
\begin{source}
Class, Aug 07
\end{source}
\end{samepage}

\begin{samepage}
\begin{ex}
Let $X$ be any set with more than one element. 
Show that the topology $\tau = \left\{ \varnothing, X \right\}$ is not metrizable.
\end{ex}
\begin{source}
Class, Aug 07
\end{source}
\end{samepage}

\begin{samepage}
\begin{ex}
Let $X = \mathbb{R} ^n$. \\
For any $x = \left (x_1, x_2, ..., x_n\right ) \in \mathbb{R} ^n$ and 
$y = \left (y_1, y_2, ..., y_n\right ) \in \mathbb{R} ^n$, define 
\begin{enumerate}
    \item $ d_1 \left (x,y\right ) = \sqrt { \sum_{i = 1}^n \left (x_i - y_i\right ) ^2 } $
    \item $ d_2 \left (x,y\right ) = \sum_{i = 1}^n |x_i - y_i| $
    \item $ d_3 \left (x,y\right ) = \displaystyle\max \left\{ | x_i - y_i |  :  1 \leq i \leq n \right\} $
\end{enumerate}
Show that $d_1$, $d_2$, $d_3$ all define the same topology on $\mathbb{R} ^n$.
\end{ex}
\begin{source}
Class, Aug 09
\end{source}
\end{samepage}

\begin{samepage}
\begin{ex}
Let $\left (X, \tau\right )$ be a topological space and $A \subseteq X$.
Suppose for each $x \in A$, there exists an open set $U$ in $X$ such that $x \in U \subseteq A$. Show that $A$ is open in $X$.
\end{ex}
\begin{source}
Problem Sheet 01, Q1
\end{source}
\end{samepage}

\begin{samepage}
\begin{ex}
Let $X$ be a set, and let $ \left\{ \tau_\alpha \right\}_{ \alpha \in I } $ be a collection of topologies on $X$. 
Show that $ \displaystyle\bigcap_{\alpha \in I} \tau_\alpha $ is a topology on $X$.
\end{ex}
\begin{source}
Problem Sheet 01, Q2
\end{source}
\end{samepage}

\begin{samepage}
\begin{ex}
Show by an example that a union of two topologies on a set $X$ may not be a topology.
\end{ex}
\begin{source}
Problem Sheet 01, Q3
\end{source}
\end{samepage}

\begin{samepage}
\begin{ex}
    \begin{enumerate}
        \item[]
        \item Consider $\left (X, \tau\right )$ where $\tau$ is the discrete topology. Then show that the collection of closed sets in $\left (X, \tau\right )$ is the power set $P\left (X\right )$
        \item Consider $\left (X, \tau\right )$ where $\tau$ is the co-finite topology. Then show that the collection of closed sets in $\left (X, \tau\right )$ is $ \left\{ A \subseteq X  :  A \text{ is finite } \right\} \cup \left\{ X \right\} $.
    \end{enumerate}
\end{ex}
\begin{source}
Class, Aug 11
\end{source}
\end{samepage}

\begin{samepage}
\begin{ex}
 Given a topological space $\left (X, \tau\right )$, show that:
    \begin{enumerate}
        \item $\varnothing$ and $X$ are closed in $X$
        \item An intersection of any collection of closed sets is closed in $X$
        \item A union of a finite collection closed sets in $X$ is closed in $X$
    \end{enumerate}
\end{ex}
\begin{source}
Class, Aug 11
\end{source}
\end{samepage}

\begin{samepage}
\begin{ex}
Consider $\left (\mathbb{R}, \tau\right )$ where $\tau$ is the usual topology on $\mathbb{R}$.
Find interior $^\mathrm{o} o_f A$ where:
    \begin{enumerate}
        \item $A = \left (0,1\right )$
        \item $A = \mathbb{Q}$
    \end{enumerate}
\end{ex}
\begin{source}
Class, Aug 11
\end{source}
\end{samepage}

\begin{samepage}
\begin{ex}
Consider $\left (\mathbb{R}, \tau\right )$ where $\tau$ is the discrete topology on $\mathbb{R}$.
Find interior $\mathbb{Q} ^\mathrm{o}$
\end{ex}
\begin{source}
Class, Aug 11
\end{source}
\end{samepage}

\begin{samepage}
\begin{ex}
Consider $\left (\mathbb{R}, \tau\right )$ where $\tau$ is the indiscrete topology on $\mathbb{R}$.
Find interior $\mathbb{Q} ^\mathrm{o}$
\end{ex}
\begin{source}
Class, Aug 11
\end{source}
\end{samepage}

\begin{samepage}
\begin{ex}
Consider $\left (\mathbb{R}, \tau\right )$ where $\tau$ is the co-finite topology on $\mathbb{R}$.
Find interior $^\mathrm{o} o_f A$ where $A = \left (0,1\right )$.
\end{ex}
\begin{source}
Class, Aug 11
\end{source}
\end{samepage}

\begin{samepage}
\begin{ex}
Let $\left (X,d\right )$ be a topological space and $S \subseteq X$. Then show that $S ^\mathrm{o}$ is the largest open set in $X$ that is contained in $S$.
\end{ex}
\begin{source}
Class, 11 Aug
\end{source}
\end{samepage}

\begin{samepage}
\begin{ex}
Let $X$ be a topological space and $S \subseteq X$. Then show that $S$ is open if and only if $S = S ^\mathrm{o}$.
\end{ex}
\begin{source}
Class, Aug 11
\end{source}
\end{samepage}

\begin{samepage}
\begin{ex}
    For subsets $A$ and $B$ and $A_\alpha$ of a topological space $\left (X, \tau\right )$ show that:
    \begin{enumerate}
        \item $A \subseteq B \implies A ^\mathrm{o} \subseteq B ^\mathrm{o}$
        \item $\left (A \cap B\right ) ^\mathrm{o} = A ^\mathrm{o} \cap B ^\mathrm{o}$
        \item $A ^\mathrm{o} \cup B ^\mathrm{o} \subseteq \left (A \cup B\right ) ^\mathrm{o}$
        \item $\left (\cap A_\alpha\right ) ^\mathrm{o} \subseteq \cap \left (A_\alpha\right ) ^\mathrm{o}$
    \end{enumerate}
\end{ex}
\begin{source}
    Problem Sheet 01, Q6
\end{source}
\end{samepage}


\begin{samepage}
\begin{ex}
Let $\tau$ be a topology on $X$ consisting of four sets $ \tau = \left\{ X, \phi, A, B \right\} $, where $A, B$ are non-empty, distinct proper subsets of $X$.
What conditions must $A$ and $B$ satisfy?
\end{ex}
\begin{source}
Schaum's P73, Q3
\end{source}
\end{samepage}

\begin{samepage}
\begin{ex}
Let $f \colon X \to Y$ be a function from a non-empty set $X$ into a topological space $\left (Y, \sigma\right )$.
Furthermore, let $\tau$ be the class of inverses of open subsets of $Y$:
$\tau = \left\{ f^{-1} \left (G\right )  :  G \in \sigma \right\} $
Show that $\tau$ is a topology on $X$.
\end{ex}
\begin{source}
Schaums, P74, Q5
\end{source}
\end{samepage}

\begin{samepage}
\begin{ex}
Let $A$ be a subset of a topological space in $X$ with the property that each point $p \in A$ belongs to an open set $G_p$ contained in $A$. 
Then prove that $A$ is open.
\end{ex}
\begin{source}
Schaums, P74, Q8
\end{source}
\end{samepage}

\begin{samepage}
\begin{ex}
Let $\tau$ be the class of subsets of $\mathbb{R}$ consisting of $\mathbb{R}$, $\varnothing$ and all open infinite intervals $E_a = \left (a, \infty\right )$ with $a \in \mathbb{R}$.
Show that $\tau$ is a topology on $\mathbb{R}$.
\end{ex}
\begin{source}
Schaum's P75, Q9
\end{source}
\end{samepage}

\begin{samepage}
\begin{ex}
Let $S$ be a subbasis for a topology on $X$. Show that the topology generated by $S$ is equal to the intersection of all topologies on $X$ that contains $S$.
\end{ex}
\begin{source}
Problem Sheet 01, Q5
\end{source}
\end{samepage}

\begin{samepage}
\begin{ex}
Show that a subset $S$ of a topological space $X$ is closed if and only if $S' \subseteq S$, that is $S$ contains all its limit points.
\end{ex}
\begin{source}
Problem Sheet 03, Q1
\end{source}
\end{samepage}

\begin{samepage}
\begin{ex}
Let $X = \mathbb{R}$ with the co-finite topology. Show that $\mathbb{N}$ is a dense subset of $\mathbb{R}$. Is $\mathbb{N}$ dense in $\mathbb{R}$ with the usual topology?
\end{ex}
\begin{source}
Problem Sheet 03, Q2
\end{source}
\end{samepage}

\begin{samepage}
\begin{ex}
Find the closure of $A = \left\{ \frac{1}{n}  :  n \in \mathbb{N} \right\}$ in $\mathbb{R}, \mathbb{R}_l, \mathbb{R}_K$.
\end{ex}
\begin{source}
Problem Sheet 03, Q3
\end{source}
\end{samepage}

\begin{samepage}
\begin{ex}
Let $X = \left\{ a, b, c, d \right\}$ Construct a non-discrete topology on $X$ such that $\left\{ a, b \right\}$ is both closed and open in $X$.
\end{ex}
\begin{source}
Problem Sheet 03, Q4
\end{source}
\end{samepage}

\begin{samepage}
\begin{ex}
Let $X = \left\{a, b, c, d \right\}$. Construct a non-discrete topology on $X$ such that $\left\{ a, b \right\}$ is neither closed nor open in $X$.
\end{ex}
\begin{source}
Problem Sheet 03, Q5
\end{source}
\end{samepage}

\begin{samepage}
\begin{ex}
Let $A, B, A_\alpha$ denote subsets of a topological space $X$. Show the following:
\begin{enumerate}
\item If $A \subseteq B$ then $\bar{A} \subseteq \bar{B}$
\item $\overline{ A \cap B } \subseteq \bar{A} \cap \bar{B}$
\item $\overline{ A \cup B } = \bar{A} \cup \bar{B}$
\item $\overline{ \cup A_\alpha } \subset \overline{ A_\alpha }$
\end{enumerate}
Also give examples where the equality in point 2. and 4. fail.
\end{ex}
\begin{source}
Problem Sheet 03, Q6
\end{source}
\end{samepage}

\begin{samepage}
\begin{ex}
Show that if $U$ is open in $X$ and $A$ is closed in $X$, then $U \setminus A$ is open in $X$. What about $A \setminus U$?
\end{ex}
\begin{source}
Problem Sheet 03, Q7
\end{source}
\end{samepage}

\begin{samepage}
\begin{ex}
    Let $X = \mathbb{R}$ and $\tau_1 = P\left (X\right )$ and $\tau_2 = \text{ usual topology }$. \\
    Does $\beta = \left\{ \left (a,b\right )  :  a,b \in \mathbb{R}, a < b \right\}$ form a basis for $\tau_1$ and $\tau_2$?
\end{ex}
\begin{source}
    Class, Aug 14
\end{source}
\end{samepage}

\begin{samepage}
\begin{ex}
    Let $\beta$ be a basis for a topology on $X$. Show that the topology generated by $\beta$ is equal to the intersection of all topologies on $X$ that contains $\beta$.
\end{ex}
\begin{source}
Problem Sheet 01, Q4
\end{source}
\end{samepage}

\begin{samepage}
\begin{ex}
    For $\mathbb{R}$ with the usual topology, is $S = \left\{ \left (- \infty, a\right )  :  a \in \mathbb{R} \right\} \cup \left\{ \left (b, \infty\right )  :  b \in \mathbb{R} \right\}$ a basis? Is it a subbasis?
\end{ex}
\begin{source}
Class, Aug 14
\end{source}
\end{samepage}

\begin{samepage}
\begin{ex}
Let $\left (X, \tau\right )$ be a topological space and $\beta$ be a collection of open sets in $\left (X, \tau\right )$. 

Show that $\beta$ is a basis for the topology $\tau$ on $X$ if and only if every open set $U$ in $\left (X, \tau\right )$ can be written as a union of members in $\beta$.
\end{ex}
\begin{source}
Class, Aug 14
\end{source}
\end{samepage}

\begin{samepage}
\begin{ex}
    If $X$ is a set, then show that a basis for a topology on $X$ is a collection $\beta$ of subsets of $X$ such that
    \begin{itemize}
        \item for every $x \in X$ there exists $B \subset \beta$ containing $x$.
        \item if $x \in B_1 \cap B_2$, for some $B_1, B_2 \in \beta$, then there exists $B_3 \in \beta$ such that $x \in B_3 \subseteq B_1 \cap B_2$.
    \end{itemize}
\end{ex}
\begin{source}
    Class, Aug 16
\end{source}
\end{samepage}

\begin{samepage}
\begin{ex}
Show that the topology generated by $\beta = \left\{ \left (a,b\right )  :  a,b \in \mathbb{R}, a < b \right\}$ is the usual topology.
\end{ex}
\begin{source}
Class, Aug 16
\end{source}
\end{samepage}

\begin{samepage}
\begin{ex}
Show that $\beta = \left\{ \left [a, b\right )  :  a, b \in \mathbb{R}, a < b \right\}$ is a basis for a topology on $\mathbb{R}$.
\end{ex}
\begin{source}
Class, Aug 16
\end{source}
\end{samepage}

\begin{samepage}
\begin{ex}
Show that the lower limit topology on $\mathbb{R}$ is strictly finer than the usual topology on $\mathbb{R}$.
\end{ex}
\begin{source}
Class, Aug 16
\end{source}
\end{samepage}

\begin{samepage}
\begin{ex}
    Let $K = \left\{ \frac{1}{n}  :  n \in \mathbb{N} \right\}$. Consider the collection $\beta = \left\{ \left (a, b\right ) \setminus K  :  a,b \in \mathbb{Q}, a < b \right\} \cup \left\{ \left (a,b\right )  :  a,b \in \mathbb{R} a < b \right\}$.  Show that the $\beta$ is a basis for a topology on $\mathbb{R}$.
\end{ex}
\begin{source}
Class, Aug 18
\end{source}
\end{samepage}

\begin{samepage}
\begin{ex}
Show that the $k$-topology on $\mathbb{R}$ is strictly finer than the usual topology.
\end{ex}
\begin{source}
Class, Aug 18
\end{source}
\end{samepage}

\begin{samepage}
\begin{ex}
Show that the lower limit topology and the $k$ topology on $\mathbb{R}$ are incomparable.
\end{ex}
\begin{source}
Class, Aug 18
\end{source}
\end{samepage}

\begin{samepage}
\begin{ex}
Let $S$ be a subbasis for a topology on $X$. Then show that 
\\ $\tau = \{ U \subseteq X  :  $ for every $ x \in U $ there exists a finite number of members $ S_1, S_2, ..., S_n \in S $ such that $ x \in S_1 \cap S_2 \cap ... \cap S_n \subseteq U \}$ is a topology generated by the subbasis $S$.
\end{ex}
\begin{source}
Class, Aug 18
\end{source}
\end{samepage}

\begin{samepage}
\begin{ex}
Let $\beta$ be a basis for a topology on $X$. Show that the topology generated by $\beta$ is equal to the intersection of all topologies on $X$ that contains $\beta$.
\end{ex}
\begin{source}
Class, Aug 18
\end{source}
\end{samepage}

\begin{samepage}
\begin{ex}
Show that the collection $\beta = \left\{ \left (a,b\right )  :  a < b, a \text{ and } b \in \mathbb{Q} \right\}$ is a basis for a topology on $\mathbb{R}$ and the topology generated by the basis is the usual topology on $\mathbb{R}$.
\end{ex}
\begin{source}
Problem Sheet 01, Q7
\end{source}
\end{samepage}

\begin{samepage}
\begin{ex}
    Show that the collection $\beta = \left\{ \left [a,b\right )  :  a < b \text{ and } a, b \in \mathbb{Q} \right\}$ is a basis for a topology on $\mathbb{R}$ and the topology generated by this basis is different from the lower limit topology on $\mathbb{R}$.
\end{ex}
\begin{source}
Problem Sheet 01, Q8
\end{source}
\end{samepage}

\begin{samepage}
\begin{ex}
Show that if $J$ is a directed set and $K$ is cofinal in $J$, then $K$ is a directed set.
\end{ex}
\begin{source}
Class, Aug 23
\end{source}
\end{samepage}

\begin{samepage}
\begin{ex}
Show that a subnet of a convergent net is convergent.
\end{ex}
\begin{source}
Class, Aug 23
\end{source}
\end{samepage}

\begin{samepage}
\begin{ex}
Let $X$ be an infinite set and $\tau$ be the co-finite topology. Then find the closure of $X$ (?)
\end{ex}
\begin{source}
Class, Aug 23
\end{source}
\end{samepage}

\begin{samepage}
\begin{ex}
Show that $\bar{S} = S' \cup S$.
\end{ex}
\begin{source}
Class, Aug 23
\end{source}
\end{samepage}

\begin{samepage}
\begin{ex}
Let $\left (X, \tau\right )$ be a metrizable space and $S \subseteq X$. Then show that $x \in \bar{S}$ if and only if there exists a sequence $\left\langle x_n\right\rangle $ in $S$ such that $x_n \to x$.
\end{ex}
\begin{source}
Class, Aug 28
\end{source}
\end{samepage}

\begin{samepage}
\begin{ex}
Let $X = \mathbb{R}$ and $\tau$ be the co-countable topology. Consider $S = \mathbb{R} \setminus \left\{ 0 \right\}$.
Does there exist a sequence $\left\langle  x_n \right\rangle $ in $S$ that such that $x_n \to 0$?
\end{ex}
\begin{source}
Class, Aug 28
\end{source}
\end{samepage}

\begin{samepage}
\begin{ex}
Let $X$ be a topological space and $S \subseteq X$. Then show that $x \in \bar{S}$ if and only if there exists a net $\left\langle  x_\lambda \right\rangle $ in $S$ such that $ x_\lambda \to x$
\end{ex}
\begin{source}
Class, Aug 28
\end{source}
\end{samepage}

\begin{samepage}
\begin{ex}
    Let $\beta$ be a basis for the topology on $X$. Show that a net $\left (x_\lambda\right )$ converges to $x$ in $X$ if for every $B \in \beta$, there exists $n_0 \in \mathbb{N}$ such that $x_n \in B \;\forall n \geq n_0$.
\end{ex}
\begin{source}
Problem Sheet 02, Q1
\end{source}
\end{samepage}

\begin{samepage}
\begin{ex}
In the space $\mathbb{R}$ with the co-finite topology, to what point or points does the sequence $\frac{1}{n}$ converge?
\end{ex}
\begin{source}
Problem Sheet 02, Q2
\end{source}
\end{samepage}

\begin{samepage}
\begin{ex}
Show that a collection $\mathcal{A}$ of subsets of $S$ that is closed under finite intersection, partially ordered by the reverse set inclusion (that is, for $A, B \in \mathcal{A}$, $A \leq B$ if $A \supseteq B$) is a directed set. Here for $A,B \in \mathcal{A}$, $A \cap B \in \mathcal{A}$ and $A \supseteq A \cap B$ and $B \supseteq A \cap B$.
\end{ex}
\begin{source}
Problem Sheet 02, Q3
\end{source}
\end{samepage}

\begin{samepage}
\begin{ex}
Show that the collection $\mathcal{F}$ of all closed subsets of a space $X$, partially ordered by the set inclusion is a directed set.
\end{ex}
\begin{source}
Problem Sheet 02, Q4
\end{source}
\end{samepage}

\begin{samepage}
\begin{ex}
Show that if a net $f$ converges to $x$ in a space $X$, so does any subnet of $f$.
\end{ex}
\begin{source}
Problem Sheet 02, Q5
\end{source}
\end{samepage}

\begin{samepage}
\begin{ex}
Show that a convergent sequence in a discrete topological space is eventually constant.
\end{ex}
\begin{source}
Problem Sheet 02, Q6
\end{source}
\end{samepage}

\begin{samepage}
\begin{ex}
Show that every net in a indiscrete topological space converges to every point of $X$.
\end{ex}
\begin{source}
Problem Sheet 02, Q7
\end{source}
\end{samepage}

\begin{samepage}
\begin{ex}
Let $X = \left\{ a, b, c \right\}$. Construct a topology $\tau$ on $X$ such that there exists a sequence in $\left (X, \tau\right )$ that converges to $a$ and $b$ both but does not converge to $c$.
\end{ex}
\begin{source}
Problem Sheet 02, Q8
\end{source}
\end{samepage}

\begin{samepage}
\begin{ex}
Let $\left (X, d\right )$ be a metric space and $\left (x_n\right )$ be a sequence in $X$. Then $x_n \to x$ if and only if for every open set $U$ in $X$ containing $x$, there exists $n_0 \in \mathbb{N}$ such that $x_n \in U \;\forall n \geq n_0$.
\end{ex}
\begin{source}
Class, Aug 21
\end{source}
\end{samepage}

\begin{samepage}
\begin{ex}
Let $X = \mathbb{R}$ and $\tau$ be the indiscrete topology. 
Let $x_n = \frac{1}{n}$ be a sequence in $\left (X, \tau\right )$.
Find if the sequence $\left (x_n\right )$ converges, and the point to where it converges.
\end{ex}
\begin{source}
Class, Aug 21
\end{source}
\end{samepage}

\begin{samepage}
\begin{ex}
Let $X$ be a discrete topological space. Show that every convergent sequence in $X$ is eventually constant.
\end{ex}
\begin{source}
Class, Aug 21
\end{source}
\end{samepage}

\begin{samepage}
\begin{ex}
Let $X$ and $\tau$ be the co-countable topological space. Show that every convergent sequence in $X$ is eventually constant.
\end{ex}
\begin{source}
\end{source}
\end{samepage}

\begin{samepage}
\begin{ex}
Show that in a topological space $\left (X, \tau\right )$, every subsequence of a convergent sequence is convergent.
\end{ex}
\begin{source}
Class, Aug 21
\end{source}
\end{samepage}

\begin{samepage}
\begin{ex}
Let $S$ be any set. Consider $P\left (S\right ) = $ the power set of $S$. Define $A \leq B$ if $A \subseteq B$. Verify that $\leq$ is a partial order relation.
Verify the same when $A \leq B$ when $A \supseteq B$.
\end{ex}
\begin{source}
Class, Aug 21
\end{source}
\end{samepage}

\begin{samepage}
\begin{ex}
Let $X$ be a topological space and $S \subseteq X$. Then $S ^\mathrm{o}$ is the largest open set contained in $S$.
\end{ex}
\begin{source}
Class, Sept 01
\end{source}
\end{samepage}

\begin{samepage}
\begin{ex}
Let $\left (X, \tau_1\right )$ and $\left (Y, \tau_2\right )$ be two topological spaces. Show that the cartesian product $\mathcal{B} = \tau_1 \times \tau_2 = \left\{ u \times v  :  u \in \tau_1, v \in \tau_2 \right\}$ will not be a topology on $X \times Y$.
Verify that $\mathcal{B}$ is a basis for the product topology on $X \times Y$.
\end{ex}
\begin{source}
Class, Sept 01
\end{source}
\end{samepage}

\begin{samepage}
\begin{ex}
Is the usual topology on $\mathbb{R} ^2$ different from the product topology on $\mathbb{R} \times \mathbb{R}$.
\end{ex}
\begin{source}
Class, Sept 01
\end{source}
\end{samepage}

\begin{samepage}
\begin{ex}
Let $\left (X, \tau\right )$ and $\left (Y, \sigma\right )$ be two topological spaces. Let $\mathcal{B}$ be a basis for $\tau$ and $\mathcal{C}$ be a basis for $\sigma$. 
Then show that the collection $\mathcal{D} = \left\{ U \times V  :  U \in \mathcal{B}, V \in \mathcal{C} \right\}$ is a basis for the product topology on
 $X \times Y$.
\end{ex}
\begin{source}
Class, Sept 01
\end{source}
\end{samepage}

\begin{samepage}
\begin{ex}
[[Stub for projection maps and one theorem (before axiom of choice) ]]
\end{ex}
\begin{source}
Class, Sept 01
\end{source}
\end{samepage}

\end{document}

