\documentclass[12pt,twoside]{report}
\usepackage[utf8]{inputenc}
\usepackage{amsmath,amsfonts,graphicx,amsthm}
\usepackage{epsfig,amssymb}
\usepackage{caption}
\usepackage{fancyhdr}
\usepackage{lipsum}
\usepackage{thmtools}
\usepackage{enumitem}
\usepackage[top=1.25in, bottom=1.25in,left=1.5in, right=1.25in]{geometry}

\newtheorem{defn}{Definition}
\newtheorem{note}{Note}
\newtheorem{notation}[subsection]{Notation}
\newtheorem{thm}{Theorem}
\newtheorem{corollary}[subsection]{Corollary}
\newtheorem{eg}[subsection]{Example}
\newtheorem{lemma}[subsection]{Lemma}
\newtheorem{result}[subsection]{Result}
\declaretheoremstyle[bodyfont=\normalfont]{normalbody}
\declaretheorem[style=normalbody,name=]{ex}
\newtheorem{remark}[subsection]{Remark}
\newtheorem{prop}[subsection]{Proposition}
\newenvironment*{ans}{\textbf{ans.}\space\em\\}{\par}
\newenvironment*{source}{\hfill\scriptsize\textbf{Source.}\space}{\par}

\declaretheoremstyle[headfont=\normalfont]{normalhead}

\makeatletter
\newcommand*{\rom}[1]{\expandafter\@slowromancap\romannumeral #1@}
\makeatother

\widowpenalties 1 10000
\raggedbottom
\setlength{\parindent}{0pt}
\renewcommand{\chaptername}{}

\setlength{\oddsidemargin}{36pt}
\setlength{\evensidemargin}{36pt}

\begin{document}

\tableofcontents
\newpage
\pagenumbering{arabic}

\chapter{Definitions and Theory}
\line(1,0){360}

\section{Topological Spaces}

\begin{defn}
A \textbf{topology} on a set $X$ is a collection $\tau$ of subsets of $X$ satisfying:
    \begin{enumerate}
        \item $\varnothing, X \in  \tau$
        \item  An intersection of finite subcollections of $\tau$ is in $\tau$
        \item A union of any subcollection of $\tau$ is in $\tau$
    \end{enumerate}
The ordered pair $\left (X, \tau\right )$ is called a \textbf{topological space}.
\end{defn}


\begin{defn}
    Let $\left (X, \tau\right )$ be a topological space. An \textbf{open subset} of $X$ is a member of $\tau$.
\end{defn}

\begin{defn}
Let $\tau$ and $\sigma$ be two topologies on a set $X$. We say that $\tau$ is \textbf{weaker} (or smaller, coarser) than $\sigma$ if $T \subseteq \sigma$. In this case, $\sigma$ is then said to be \textbf{stronger} (or larger, finer) than $\tau$.
\end{defn}

\begin{defn}
Let $X$ be any set. The collection $\tau = P\left (X\right )$ is a topology on $X$ and is called the \textbf{discrete topology} on $X$. Here $\left (X, \tau\right )$ is called the \textbf{discrete topological space}.
\end{defn}

\begin{defn}
Let $X$ be any set. The collection $\tau = \left\{ \varnothing, X \right\}$ is called the \textbf{indiscrete topology} on $X$. Here $\left (X, \tau\right )$ is called the \textbf{indiscrete topology}.
\end{defn}

\begin{defn}
    Let $X$ be any set. The collection $\\tau = \left\{ A \subseteq X  :  X \setminus A \text{ is finite } \right\} \cup \left\{ \varnothing \right\}$ is called the \textbf{co-finite topology}.
\end{defn}

\begin{defn}
    Let $X$ be any set. The collection $\\tau = \left\{ A \subseteq X  :  X \setminus A \text{ is countable } \right\} \cup \left\{ \varnothing \right\}$ is called the \textbf{co-finite topology}.
\end{defn}



\begin{defn}
A topology $\tau$ on a set $X$ is said to be \textbf{metrizable} if there exists a metric $d$ on $X$ such that the topology $\tau_d$ generated by the metric $d$ coincides with $\tau$.
\end{defn}

\begin{defn}
Two metrics defined on a set $X$ are said to be \textbf{equivalent} if they generate the same topology. 
In other words, $d_1$ and $d_2$ are equivalent if the collection of open sets in $\left (X, d_1\right )$ and $\left (X, d_2\right )$ are the same.
\end{defn}

\begin{defn}
The topology generated by the Euclidean metric on $\mathbb{R} ^n$ is called the \textbf{usual topology} on $\mathbb{R} ^n$.
For $Y \subseteq \mathbb{R} ^n$, the topology generated by the Euclidean metric is called the usual topology on $Y$.
\end{defn}

\begin{defn}
By a \textbf{neighbourhood} of a point $x$ in a topological space $\left (X, \tau\right )$, we mean an open set containing $x$.
\end{defn}

\begin{defn}
A subset $A$ of a topological space $\left (X, \tau\right )$ is said to be \textbf{closed} if $X \setminus A$ is open in $X$, that is $X \setminus A \in \tau$
\end{defn}

\begin{prop}
    \begin{itemize}
        \item[]
        \item  $\varnothing$ and $X$ are closed in $X$
        \item An intersection of any collection of closed sets is closed in $X$
        \item A union of a \textit{finite} collection of closed sets in $X$ is closed in $X$
    \end{itemize}
\end{prop}

\begin{defn}
Let $X$ be a topological space and $A \subseteq X$. A point $x \in X$ is said to be an \textbf{interior point} of $A$ if there exists a neighbourhood $U$ of $X$ such that $U \subseteq A$\\ or in other words, if there exists an open set $U$ in $X$ such that $x \in U \subseteq A$. 

The set of all interior points of $A$ is called the \textbf{interior} of $A$ and is denoted $A^\mathrm{o}$.
\end{defn}

\begin{thm}
    Let $\left (X, d\right )$ be a topological space and $S \subseteq X$. Then $S^\mathrm{o}$ is the largest open set in $X$ that is contained in $S$.
\end{thm}

\begin{thm}
    Let $X$ be a topological space and $S \subseteq X$ Then $S$ is open if and only if $S = S^\mathrm{o}$.
\end{thm}

\section{Bases and Subbases}

\begin{defn}
    Let $X$ be a topological space. A collection $\beta$ of open subsets of $X$ is said to be a \textbf{basis} for the topology on $X$ if for every open set $U$ in $X$ and $x \in U$, there exists a $B \in \beta$ such that $x \in B \subseteq U$.

    Members of $\beta$ are called \textbf{basis open sets} corresponding to basis $\beta$.
\end{defn}

\begin{note}
    For any topological space $\left (X, \tau\right )$, $\tau$ is a basis for $\tau$.
\end{note}

\begin{note}
    In $\mathbb{R}$ the set $\left\{ \left (x - \varepsilon, x + \varepsilon\right )  :  x \in \mathbb{R}\text{ and }\varepsilon >0 \right\}$ is a basis for the usual topology.
\end{note}

\begin{note}
    For $\left (\mathbb{R}, \tau\right )$, where $\tau$ is the discrete topology, $\beta = \left\{ \left\{ x \right\}  :  x \in \mathbb{R} \right\}$ is a basis for $\tau$ on $\mathbb{R}$.
\end{note}

\begin{note}
    In $\mathbb{R}$ the set $\left\{ \left (x - \frac{1}{n}, x + \frac{1}{n}\right )  :  x \in \mathbb{R}\text{ and }n \in \mathbb{N} \right\}$ is a basis for the usual topology.
\end{note}

\begin{note}
    In $\mathbb{R}$ the set $\left\{ \left (a, b\right )  :  a,b \in \mathbb{Q}\text{ and }a < b \right\}$ is a basis for the usual topology. This is a countable basis for $\mathbb{R}$ with the usual topology.
\end{note}

\begin{note}
    Let $\left (X, \tau\right )$ be a metrizable space. Then $\beta = \left\{ B\left (x, \varepsilon\right )  :  x \in X, \varepsilon >0 \right\}$ is a basis for $\tau$.
\end{note}

\begin{defn}
    Let $\left (X, \tau\right )$ be a topological space. A collection $S \subseteq \tau$ is said to be a \textbf{subbasis} for the topology $\tau$ if for every open set $U$ in $\tau$ and $x \in U$, there exists a finite subcollection $\left\{ s_1, s_2, ..., s_n \right\}$ in $S$ such that $x \in \displaystyle\bigcap_{i = 1}^n S_i \subseteq U$

    Members of $S$ are called \textbf{subbasis open sets} corresponding to $S$.
\end{defn}

\begin{note}
    For a topological space $\left (X, \tau\right )$ a collection $S$ is a subbasis for the topology $\tau$ if and only if the collection of all finite intersections of members in $S$ forms a basis for the topology $\tau$.
\end{note}

\begin{thm}
    Let $\left (X, \tau\right )$ be a topological space and $\beta$ be a collection of open sets in $\left (X, \tau\right )$.

    Then $\beta$ is a basis for the topology $\tau$ on $X$ if and only if every open set $U$ in $\left (X, \tau\right )$ can be written as a union of members in $\beta$.
\end{thm}

\chapter{Exercises}
\line(1,0){360}

\begin{samepage}
\begin{ex}
Let $\left (X, d\right )$ be a metric space and $\tau_d$ be the collection of all open subsets of $X$. 
Show that $\tau_d$ is a topology on $X$
\vspace{0.5cm}
\textit{This is called the topology on $X$ generated by $d$}
\end{ex}
\begin{source}
Class, Aug 07
\end{source}
\end{samepage}

\begin{samepage}
\begin{ex}
Let $X$ be any set and $A \subseteq X$. Then show that 
$\tau = \left\{ \varnothing, A , X \right\}$ is a topology on $X$.
\end{ex}
\begin{source}
Class, Aug 07
\end{source}
\end{samepage}

\begin{samepage}
\begin{ex}
Let $X$ be any set. Let 
$\tau = \left\{ A \subseteq X  :  X \setminus A \text{ is finite } \right\} \cup \left\{ \varnothing \right\} $
Then show that $\tau$ is a topology on $X$.
\end{ex}
\begin{source}
Class, Aug 07
\end{source}
\end{samepage}

\begin{samepage}
\begin{ex}
Let $X$ be any set and 
$\tau = \left\{ A \subseteq X  :  X \setminus A \text{ is countable } \right\} \cup \left\{ \varnothing \right\} $.
Then show that $\tau$ is a topology on $X$.
\end{ex}
\begin{source}
Class, Aug 07
\end{source}
\end{samepage}

\begin{samepage}
\begin{ex}
Show that every discrete topological space is metrizable
\end{ex}
\begin{source}
Class, Aug 07
\end{source}
\end{samepage}

\begin{samepage}
\begin{ex}
Let $X$ be any set with more than one element. 
Show that the topology $\tau = \left\{ \varnothing, X \right\}$ is not metrizable.
\end{ex}
\begin{source}
Class, Aug 07
\end{source}
\end{samepage}

\begin{samepage}
\begin{ex}
Let $X = \mathbb{R} ^n$. \\
For any $x = \left (x_1, x_2, ..., x_n\right ) \in \mathbb{R} ^n$ and 
$y = \left (y_1, y_2, ..., y_n\right ) \in \mathbb{R} ^n$, define 
\begin{enumerate}
    \item $ d_1 \left (x,y\right ) = \sqrt { \sum_{i = 1}^n \left (x_i - y_i\right ) ^2 } $
    \item $ d_2 \left (x,y\right ) = \sum_{i = 1}^n |x_i - y_i| $
    \item $ d_3 \left (x,y\right ) = \displaystyle\max \left\{ | x_i - y_i |  :  1 \leq i \leq n \right\} $
\end{enumerate}
Show that $d_1$, $d_2$, $d_3$ all define the same topology on $\mathbb{R} ^n$.
\end{ex}
\begin{source}
Class, Aug 09
\end{source}
\end{samepage}

\begin{samepage}
\begin{ex}
Let $\left (X, \tau\right )$ be a topological space and $A \subseteq X$.
Suppose for each $x \in A$, there exists an open set $U$ in $X$ such that $x \in U \subseteq A$. Show that $A$ is open in $X$.
\end{ex}
\begin{source}
Problem Sheet 01, Q1
\end{source}
\end{samepage}

\begin{samepage}
\begin{ex}
Let $X$ be a set, and let $ \left\{ \tau_\alpha \right\}_{ \alpha \in I } $ be a collection of topologies on $X$. 
Show that $ \displaystyle\bigcap_{\alpha \in I} \tau_\alpha $ is a topology on $X$.
\end{ex}
\begin{source}
Problem Sheet 01, Q2
\end{source}
\end{samepage}

\begin{samepage}
\begin{ex}
Show by an example that a union of two topologies on a set $X$ may not be a topology.
\end{ex}
\begin{source}
Problem Sheet 01, Q3
\end{source}
\end{samepage}

\begin{samepage}
\begin{ex}
Consider $\left (X, \tau\right )$ where $\tau$ is the discrete topology. Then the collection of closed sets is the power set $P\left (X\right )$ which coincides with the collection open sets in $\left (X, \tau\right )$.\\
Consider $\left (X, \tau\right )$ where $\tau$ is the co-finite topology. Then the collection of closed sets in $\left (X, \tau\right )$ is $ \left\{ A \in \tau  :  A \text{ is finite } \right\} \cup \left\{ X \right\} $.
\end{ex}
\begin{source}
Class, Aug 11
\end{source}
\end{samepage}

\begin{samepage}
\begin{ex}
 Given a topological space $\left (X, \tau\right )$, show that:
    \begin{enumerate}
        \item $\varnothing$ and $X$ are closed in $X$
        \item An intersection of any collection of closed sets is closed in $X$
        \item A union of a finite collection closed sets in $X$ is closed in $X$
    \end{enumerate}
\end{ex}
\begin{source}
Class, Aug 11
\end{source}
\end{samepage}

\begin{samepage}
\begin{ex}
Consider $\left (\mathbb{R}, \tau\right )$ where $\tau$ is the usual topology on $\mathbb{R}$.
Find $A^\mathrm{o}$ where:
    \begin{enumerate}
        \item $A = \left (0,1\right )$
        \item $A = \mathbb{Q}$
    \end{enumerate}
\end{ex}
\begin{source}
Class, Aug 11
\end{source}
\end{samepage}

\begin{samepage}
\begin{ex}
Consider $\left (\mathbb{R}, \tau\right )$ where $\tau$ is the discrete topology on $\mathbb{R}$.
Find $\mathbb{Q} ^\mathrm{o}$
\end{ex}
\begin{source}
Class, Aug 11
\end{source}
\end{samepage}

\begin{samepage}
\begin{ex}
Consider $\left (\mathbb{R}, \tau\right )$ where $\tau$ is the indiscrete topology on $\mathbb{R}$.
Find $\mathbb{Q} ^\mathrm{o}$
\end{ex}
\begin{source}
Class, Aug 11
\end{source}
\end{samepage}

\begin{samepage}
\begin{ex}
Consider $\left (\mathbb{R}, \tau\right )$ where $\tau$ is the co-finite topology on $\mathbb{R}$.
Find $A^\mathrm{o}$ where $A = \left (0,1\right )$.
\end{ex}
\begin{source}
Class, Aug 11
\end{source}
\end{samepage}

\begin{samepage}
\begin{ex}
Let $\left (X,d\right )$ be a topological space and $S \subseteq X$. Then show that $S ^\mathrm{o}$ is the largest open set in $X$ that is contained in $S$.
\end{ex}
\begin{source}
Class, 11 Aug
\end{source}
\end{samepage}

\begin{samepage}
\begin{ex}
Let $X$ be a topological space and $S \subseteq X$. Then show that $S$ is open if and only if $S = S ^\mathrm{o}$.
\end{ex}
\begin{source}
Class, Aug 11
\end{source}
\end{samepage}

\begin{samepage}
\begin{ex}
[ Stub for Problem Sheet 1, Q6] and remaining from Problem Sheet 1 and 2
\end{ex}
\begin{source}
\end{source}
\end{samepage}


\begin{samepage}
\begin{ex}
Let $\tau$ be a topology on $X$ consisting of four sets $ \tau = \left\{ X, \phi, A, B \right\} $, where $A, B$ are non-empty, distinct proper subsets of $X$.
What conditions must $A$ and $B$ satisfy?
\end{ex}
\begin{source}
Schaum's P73, Q3
\end{source}
\end{samepage}

\begin{samepage}
\begin{ex}
Let $f \colon X \to Y$ be a function from a non-empty set $X$ into a topological space $\left (Y, \sigma\right )$.
Furthermore, let $\tau$ be the class of inverses of open subsets of $Y$:
$\tau = \left\{ f^{-1} \left (G\right )  :  G \in \sigma \right\} $
Show that $\tau$ is a topology on $X$.
\end{ex}
\begin{source}
Schaums, P74, Q5
\end{source}
\end{samepage}

\begin{samepage}
\begin{ex}
Let $A$ be a subset of a topological space in $X$ with the property that each point $p \in A$ belongs to an open set $G_p$ contained in $A$. 
Then prove that $A$ is open.
\end{ex}
\begin{source}
Schaums, P74, Q8
\end{source}
\end{samepage}

\begin{samepage}
\begin{ex}
Let $\tau$ be the class of subsets of $\mathbb{R}$ consisting of $\mathbb{R}$, $\varnothing$ and all open infinite intervals $E_a = \left (a, \infty\right )$ with $a \in \mathbb{R}$.
Show that $\tau$ is a topology on $\mathbb{R}$.
\end{ex}
\begin{source}
Schaum's P75, Q9
\end{source}
\end{samepage}

\begin{samepage}
\begin{ex}
    Let $X = \mathbb{R}$ and $\tau_1 = P\left (X\right )$ and $\tau_2 = \text{ usual topology }$. \\
    Does $\beta = \left\{ \left (a,b\right )  :  a,b \in \mathbb{R}, a < b \right\}$ form a basis for $\tau_1$ and $\tau_2$?
\end{ex}
\begin{source}
    Class, Aug 14
\end{source}
\end{samepage}

\begin{samepage}
\begin{ex}
[Stub for Problem 4, 7, 8 from Problem Sheet 01]
\end{ex}
\begin{source}
Problem Sheet 01
\end{source}
\end{samepage}

\begin{samepage}
\begin{ex}
    For $\mathbb{R}$ with the usual topology, is $S = \left\{ \left (- \infty, a\right )  :  a \in \mathbb{R} \right\} \cup \left\{ \left (b, \infty\right )  :  b \in \mathbb{R} \right\}$ a basis? Is it a subbasis?
\end{ex}
\begin{source}
Class, Aug 14
\end{source}
\end{samepage}

\begin{samepage}
\begin{ex}
Let $\left (X, \tau\right )$ be a topological space and $\beta$ be a collection of open sets in $\left (X, \tau\right )$. 

Show that $\beta$ is a basis for the topology $\tau$ on $X$ if and only if every open set $U$ in $\left (X, \tau\right )$ can be written as a union of members in $\beta$.
\end{ex}
\begin{source}
Class, Aug 14
\end{source}
\end{samepage}

\end{document}

