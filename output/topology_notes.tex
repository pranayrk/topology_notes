\documentclass[14pt,twoside]{extreport}
\usepackage[utf8]{inputenc}
\usepackage{amsmath,amsfonts,graphicx,amsthm}
\usepackage{epsfig,amssymb}
\usepackage{caption}
\usepackage{fancyhdr}
\usepackage{lipsum}
\usepackage{thmtools}
\usepackage{enumitem}
\usepackage[margin=0.75in]{geometry}

\tolerance=1
\emergencystretch=\maxdimen
\hyphenpenalty=10000
\hbadness=10000
\widowpenalties 1 10000
\raggedbottom
\setlength{\parindent}{0pt}
\renewcommand{\chaptername}{}
\newcommand{\hhrule}{\vspace{1cm}\hrule\vspace{1cm}}


\newtheorem*{defn}{Definition}
\newtheorem*{thm}{Theorem}
\newtheorem*{note}{Note}
\newtheorem*{lemma}{Lemma}
\newtheorem*{result}{Result}
\newtheorem*{remark}{Remark}
\newtheorem*{prop}{Proposition}

\begin{document}

\newpage
\pagenumbering{arabic}

\chapter*{Topology}
\line(1,0){360}

\begin{defn}
    Let $\left (X, \tau_1\right )$ and $\left (Y, \tau_2\right )$ be two topological spaces. The collection $\mathcal{B} = \tau_1 \times \tau_2 = \left\{ u \times v  :  u \in \tau_1, v \in \tau_2 \right\}$ is a basis for a topology on $X \times Y$.
    The topology generated by $\mathcal{B}$ is called the \textbf{product topology} on $X \times Y$.
\end{defn}

\begin{thm}
    Let $\left (X, \tau\right )$ and $\left (Y, \sigma\right )$ be two topological spaces. Let $\mathcal{B}$ be a basis for $\tau$ and $\mathcal{C}$ be a basis for $\sigma$.
    Then the collection $\mathcal{D} = \left\{ U \times V  :  U \in \mathcal{B}, V \in \mathcal{C} \right\}$ is a basis for the product topology on $X \times Y$.
\end{thm}

\begin{result}
    The product topology on $\m_at_hb_b{R} \\cdot \m_at_hb_b{R}$ coincides with the usual topology on $\m_at_hb_b{R}^2$
\end{result}

\hhrule

\end{document}

