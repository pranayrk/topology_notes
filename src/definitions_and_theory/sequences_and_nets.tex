\begin{defn}
    Let $X$ be a set. A function $f from naturals to X$ is called a \textbf{sequence} in $X$.
\end{defn}

\begin{defn}
    Let $(X, Y)$ be a topological space. A sequence $(X sub n)$ is said to \textbf{converge} to $x in X$ if for every neighbourhood $U$ of $x$, there exists $n0 in naturals$ such that $xn in U for all n greater than n0$
\end{defn}

\begin{defn}
    A sequence $(xn)$ is said to be \textbf{eventually constant} if there exists $n0 in naturals$ such that $xn = x sub {n0} for all n greater than n0$.
\end{defn}

\begin{thm}
    In any topological space, every eventually constant sequence is convergent.
\end{thm}

\begin{defn}
    Let $f from naturals to X$ be a sequence in $X$. Then for any strictly increasing function $g from naturals to naturals$, the composition $f composed g from naturals to X$ is called a \textbf{subsequence} of $f$.
\end{defn}

\begin{thm}
    In a topological space $(X, tau)$, every subsequence of a convergent sequence is convergent.
\end{thm}

\begin{defn}
    Let $script A$ be a non-empty set. A relation $less than$ on a set $A$ is called a \textbf{partial order relation} if the following condition holds for all $alpha, beta, gamma$ in $A$:
    \begin{enumerate}
        \item reflexive: $alpha less than alpha$
        \item anti-symmetric: $alpha less than beta$ and $beta less than alpha$ $implies$ $alpha = beta$
        \item transitive: $alpha less than beta$ and $beta less than gamma$ $implies$ $alpha less than gamma$
    \end{enumerate}
\end{defn}

\begin{defn}
    A \textbf{directed set} $J$ is a set with a partial order relation $less than$ such that for each pair $alpha$ and $beta$ of $J$, there exists a $Y in J$ such that $alpha less than Y$ and $beta less than Y$.
\end{defn}

\begin{defn}
    A \textbf{net} in $X$ is a function $f$ from a directed set $J$ to $X$.
\end{defn}

\begin{note}
Every sequence is a net.
\end{note}

\begin{defn}
    Let $J$ be a directed set with a partial order relation '$less than$'. A subset $K$ of $J$ is said to be \textbf{cofinal} in $J$ if for each $alpha in J$, there exists $beta in K$ such that $alpha less than beta$.
\end{defn}

\begin{prop}
    If $g from naturals to naturals$ is a strictly increasing function then $g(naturals)$ is cofinal in $naturals$.
\end{prop}

\begin{thm}
    If $J$ is a directed set and $K$ is cofinal in $J$, then $K$ is a directed set.
\end{thm}

\begin{defn}
    Let $f from J to X$ be a net in $X$. If $I$ is a directed set and $g from I to J$ such that
    \begin{itemize}
        \item $i strictly less than j implies g(i) strictly less than g(j) for all i, j in I$
        \item $g(I)$ is cofinal in $J$
    \end{itemize}
    Then $f composed g from I to X$ is called a \textbf{subnet} of $f$.
\end{defn}

\begin{defn}
    The net $(X sub alpha) sub {alpha in J}$ is said to \textbf{converge} to a point $x in X$ if for every neighbourhood $U$ of $x$, there exists $alpha sub 0 in J$ such that $x sub alpha in U$ for all $alpha greater than alpha sub 0$.
\end{defn}

\hhrule 
