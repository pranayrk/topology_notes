\begin{defn}
    Let $X$ be a set. A function $f \colon \mathbb{N} \to X$ is called a \textbf{sequence} in $X$.
\end{defn}

\begin{defn}
    Let $\left (X, Y\right )$ be a topological space. A sequence $\left (X_n\right )$ is said to \textbf{converge} to $x \in X$ if for every neighbourhood $U$ of $x$, there exists $n_0 \in \mathbb{N}$ such that $x_n \in U \;\forall n \geq n_0$
\end{defn}

\begin{defn}
    A sequence $\left (x_n\right )$ is said to be \textbf{eventually constant} if there exists $n_0 \in \mathbb{N}$ such that $x_n = x_{n_0} \;\forall n \geq n_0$.
\end{defn}

\begin{thm}
    In any topological space, every eventually constant sequence is convergent.
\end{thm}

\begin{defn}
    Let $f \colon \mathbb{N} \to X$ be a sequence in $X$. Then for any strictly increasing function $g \colon \mathbb{N} \to \mathbb{N}$, the composition $f \circ g \colon \mathbb{N} \to X$ is called a \textbf{subsequence} of $f$.
\end{defn}

\begin{thm}
    In a topological space $\left (X, \tau\right )$, every subsequence of a convergent sequence is convergent.
\end{thm}

\begin{defn}
    Let $\mathcal{A}$ be a non-empty set. A relation $\leq$ on a set $A$ is called a \textbf{partial order relation} if the following condition holds for all $\alpha, \beta, \gamma$ in $A$:
    \begin{enumerate}
        \item reflexive: $\alpha \leq \alpha$
        \item anti-symmetric: $\alpha \leq \beta$ and $\beta \leq \alpha$ $\implies$ $\alpha = \beta$
        \item transitive: $\alpha \leq \beta$ and $\beta \leq \gamma$ $\implies$ $\alpha \leq \gamma$
    \end{enumerate}
\end{defn}

\begin{defn}
    A \textbf{directed set} $J$ is a set with a partial order relation $\leq$ such that for each pair $\alpha$ and $\beta$ of $J$, there exists a $Y \in J$ such that $\alpha \leq Y$ and $\beta \leq Y$.
\end{defn}

\begin{defn}
    A \textbf{net} in $X$ is a function $f$ from a directed set $J$ to $X$.
\end{defn}

\begin{note}
Every sequence is a net.
\end{note}

\begin{defn}
    Let $J$ be a directed set with a partial order relation '$\leq$'. A subset $K$ of $J$ is said to be \textbf{cofinal} in $J$ if for each $\alpha \in J$, there exists $\beta \in K$ such that $\alpha \leq \beta$.
\end{defn}

\begin{prop}
    If $g \colon \mathbb{N} \to \mathbb{N}$ is a strictly increasing function then $g\left (\mathbb{N}\right )$ is cofinal in $\mathbb{N}$.
\end{prop}

\begin{thm}
    If $J$ is a directed set and $K$ is cofinal in $J$, then $K$ is a directed set.
\end{thm}

\begin{defn}
    Let $f \colon J \to X$ be a net in $X$. If $I$ is a directed set and $g \colon I \to J$ such that
    \begin{itemize}
        \item $i < j \implies g\left (i\right ) < g\left (j\right ) \;\forall i, j \in I$
        \item $g\left (I\right )$ is cofinal in $J$
    \end{itemize}
    Then $f \circ g \colon I \to X$ is called a \textbf{subnet} of $f$.
\end{defn}

\begin{defn}
    The net $\left (X_\alpha\right )_{\alpha \in J}$ is said to \textbf{converge} to a point $x \in X$ if for every neighbourhood $U$ of $x$, there exists $\alpha_0 \in J$ such that $x_\alpha \in U$ for all $\alpha \geq \alpha_0$.
\end{defn}

\hhrule 
