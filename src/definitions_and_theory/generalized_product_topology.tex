\begin{defn}
    Let $\{ X_\alpha \}_{\alpha \in I}$ be an indexed family of topological spaces. For the product space $X =  \prod_{\alpha \in I} X_\alpha$, we take the basis $\mathscr{B} = \{ \prod_{\alpha \in I} U_\alpha | U_\alpha \text{ is open in } X_\alpha \}$. The topology generated by this basis is called the \textbf{box topology}. 
\end{defn}

\begin{note}
    The above is a direct generalization of the earlier product topology defined on product of two spaces.
\end{note}

\begin{thm}
    A subset $W$ of $X$ is open in $X$ with the box topology iff for every $x = (x_\alpha)_{\alpha \in I}$, there exists an open set $U_\alpha$ in $X_\alpha$ such that $(x_\alpha)_{\alpha \in I} \in \prod_{\alpha \in I} U_\alpha \subseteq W$
\end{thm}

\begin{defn}
Let $\{ X_\alpha \}_{\alpha \in I}$ be an indexed family of topological spaces. For the product space $X =  \prod_{\alpha \in I} X_\alpha$, we take the basis $\mathscr{C} = \{ \prod_{\alpha \in I} U_\alpha | U_\alpha \text{ is open in } X, U_\alpha = X_\alpha \text{ for all but finitely many } \alpha \in I \}$. 

    The topology generated by this basis is called the \textbf{product topology} on $X$.
\end{defn}

\begin{note}
    Unless mentioned otherwise, we usually consider the product topology on the product space $X$.
\end{note}

\begin{thm}
    The product topology on $X$ is weaker than the box topology on $X$.
\end{thm}

\begin{note}
    When the index $I$ is finite, the product topology and the box topology coincide.
\end{note}

\begin{defn}
    Let $\{ X_\alpha \}_{\alpha \in I}$ be an indexed family of topological spaces and $X = \prod_{\alpha \in I} X_\alpha$ be the product space. 

    For every $\beta \in I$, the \textbf{$\beta$-th projection map} $\Pi_\beta: X \to X_\beta$ is defined as $\Pi_\beta ( (x_\alpha)_{\alpha \in I} ) = x_\beta$
\end{defn}

\begin{thm}
Let $\{ X_\alpha \}_{\alpha \in I}$ be an indexed family of topological spaces and $X = \prod_{\alpha \in I} X_\alpha$ be the product space. 

    The collection $\mathscr{S} = \{ \Pi^{-1}_\beta (U_\beta) | U_\beta \text{ is open in } X_\beta, \beta \in I \}$is a subbasis for the product topology on $X$.
\end{thm}

\begin{thm}
Let $\{ X_\alpha \}_{\alpha \in I}$ be an indexed family of topological spaces and $X = \prod_{\alpha \in I} X_\alpha$ be the product space equipped with the product topology.

    Then a function $f: A \to X$ is continuous iff $f_\alpha = \Pi_\alpha \circ f : A \to X_\alpha$ is continuous for each $\alpha \in I$.
\end{thm}

\begin{thm}
    The topologies on $\mathbb{R}^n$ induced by the euclidean metric and the square norm are the same as the product topology on $\mathbb{R}^n$.
\end{thm}

\begin{defn}
    Consider the standard bounded metric on $\mathbb{R}$ defined as $\overline{d} = \min\{ |a-b|, 1\}$.
    Given an index set $I$ and points $x = (x_\alpha)_{\alpha \in I}$ and $y = (y_\alpha)_{\alpha \in I}$, the metric $\rho$ on $\mathbb{R}^I$ defined  as $\rho(x,y) = \sup \{ \overline{d}(x_\alpha, y_\alpha) | \alpha \in I \}$ is known as the \textbf{uniform metric} on $\mathbb{R}^I$.

    The topology generated by the uniform metric is called the \textbf{uniform topology} on $\mathbb{R}^I$.
\end{defn}

\begin{thm}
    The uniform topology on $\mathbb{R}^I$ is weaker than the box topology and stronger than the product topology. 
    All these topologies are different if $I$ is infinite.
\end{thm}

\begin{defn}
    Let $X$ and $Y$ be topological spaces and let $p: X \to Y$ be a surjective map. 

    The map $p$ is said to be a \textbf{quotient map} provided a subset $U$ of $Y$ is open in $Y$ iff $p^{-1}(U)$ is open in $X$.
\end{defn}

\begin{note}
    We can replace 'open' in the above definition with 'closed'.
\end{note}

\begin{note}
    Every quotient map is a continuous map.
\end{note}

\begin{defn}
    A map $f: X \to Y$ is said to be \textbf{open} if for each open set $U$ in $X$, $f(U)$ is open in $Y$.
    A map $f: X \to Y$ is said to be \textbf{closed} if for each closed set $C$ in $X$, $f(C)$ is closed in $Y$.
\end{defn}


\begin{note}
    A surjective continuous map which is either open or closed is a quotient map. Note, there are quotient maps which are neither open nor closed.
\end{note}

\begin{defn}
    If $X$ is a topological space and $A$ is any set, and if $p: X \to A$ is a surjective map, then there exists exactly one topology $\tau$ on $A$ relative to which $p$ is a quotient map. This topology is known as the \textbf{quotient topology} induced by $p$.
\end{defn}

\begin{defn}
    Let $X$ be a topological space and $\sim$ be an equivalence relation on $X$. 

    Define a map $p: X \to \frac{X}{\sim}$ as $p(x) = [x]$ where $[x]$ is the equivalence class of $x$ under $\sim$.

    Let $\tau$ be the quotient topology induced by $p$ on $\frac{X}{\sim}$. Then the space $(\frac{X}{\sim}, \tau)$ is called the \textbf{quotient space} of $X$.
\end{defn}


\hhrule


