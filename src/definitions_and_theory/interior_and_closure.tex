\begin{defn}
    Let $\left (X, \tau\right )$ be a topological space and $S \subseteq X$. A point $x \in X$ is called a \textbf{closure point} of $S$ if for every neighbourhood $U$ of $x$, we have $U \cap S \neq  \varnothing$.

    The set of all closure points of $S$ is called the \textbf{closure} of $A$ and is denoted $\text{cl}\left (A\right )$ or $\bar{A}$.
\end{defn}

\begin{defn}
    A point $x \in X$ is said to be a \textbf{limit point} of $S$ if for every neighbourhood of $x$, we have $\left (U \cap S\right ) \setminus \left\{ x \right\} \neq  \varnothing $.

    The set of all limit points is called the \textbf{derived set} and is denoted $S'$
\end{defn}

\begin{thm}
    Let $X$ be a topological space and $S \subseteq X$. Then $\bar{S}$ is the smallest closed set in $X$ that contains $S$.
\end{thm}

\begin{prop}
    A subset of a topological space $X$ is closed if and only if $S = \bar{S}$
\end{prop}

\begin{thm}
    Let $\left (X, \tau\right )$ be a metrizable space and $S \subseteq X$. Then $x \in \bar{S}$ if and only if there exists a sequence $\left\langle x_n\right\rangle $ in $S$ such that $x_n \to x$.
\end{thm}

\begin{thm}
    Let $X$ be a topological space and $S \subseteq X$. Then show that $x \in \bar{S}$ if and only if there exists a net $\left\langle  x_\lambda \right\rangle $ in $S$ such that $ x_\lambda \to x$
\end{thm}

\begin{defn}
        Let $X$ be a topological space and $A \subseteq X$. A point $x \in X$ is said to be an \textbf{interior point} of $A$ if there exists a neighbourhood $U$ of $X$ such that $U \subseteq A$ (or) if there exists an open set $U$ in $X$ such that $x \in U \subseteq A$.

        The set of all interior points of $A$ is called the \textbf{interior} of $A$ and is denoted $A ^\mathrm{o}$.
\end{defn}

\begin{thm}
    Let $X$ be a topological space and $S \subseteq X$. Then $S ^\mathrm{o}$ is the largest open set contained in $S$.
\end{thm}

\begin{thm}
    Let $X$ be a topological space and $S \subseteq X$. Then $S$ is open if and only if $S = S ^\mathrm{o}$.
\end{thm}

\hhrule 
