\begin{defn}
A \textbf{topology} on a set $X$ is a collection $tau$ of subsets of $X$ satisfying:
    \begin{enumerate}
        \item $empty set, X are in tau$
        \item  An intersection of finite subcollections of $tau$ is in $tau$
        \item A union of any subcollection of $tau$ is in $tau$
    \end{enumerate}
The ordered pair $(X, tau)$ is called a \textbf{topological space}.
\end{defn}


\begin{defn}
    Let $(X, tau)$ be a topological space. An \textbf{open subset} of $X$ is a member of $tau$.
\end{defn}

\begin{defn}
Let $tau$ and $sigma$ be two topologies on a set $X$. We say that $tau$ is \textbf{weaker} (or smaller, coarser) than $sigma$ if $T subset of sigma$. In this case, $sigma$ is then said to be \textbf{stronger} (or larger, finer) than $tau$.
\end{defn}

\begin{defn}
Let $X$ be any set. The collection $tau = P(X)$ is a topology on $X$ and is called the \textbf{discrete topology} on $X$. Here $(X, tau)$ is called the \textbf{discrete topological space}.
\end{defn}

\begin{defn}
Let $X$ be any set. The collection $tau = {{ empty set, X }}$ is called the \textbf{indiscrete topology} on $X$. Here $(X, tau)$ is called the \textbf{indiscrete topology}.
\end{defn}

\begin{defn}
    Let $X$ be any set. The collection $\tau = {{ A subset of X such that X minus A @@\text{ is finite }@@ }} union {{ empty set }}$ is called the \textbf{co-finite topology}.
\end{defn}

\begin{defn}
    Let $X$ be any set. The collection $\tau = {{ A subset of X such that X minus A @@\text{ is countable }@@ }} union {{ empty set }}$ is called the \textbf{co-countable topology}.
\end{defn}



\begin{defn}
A topology $tau$ on a set $X$ is said to be \textbf{metrizable} if there exists a metric $d$ on $X$ such that the topology $tau sub d$ generated by the metric $d$ coincides with $tau$.
\end{defn}

\begin{defn}
Two metrics defined on a set $X$ are said to be \textbf{equivalent} if they generate the same topology. 
In other words, $d sub 1$ and $d sub 2$ are equivalent if the collection of open sets in $(X, d sub 1)$ and $(X, d sub 2)$ are the same.
\end{defn}

\begin{defn}
The topology generated by the Euclidean metric on $reals raised to n$ is called the \textbf{usual topology} on $reals raised to n$.
For $Y subset of reals raised to n$, the topology generated by the Euclidean metric is called the usual topology on $Y$.
\end{defn}

\begin{defn}
By a \textbf{neighbourhood} of a point $x$ in a topological space $(X, tau)$, we mean an open set containing $x$.
\end{defn}

\begin{defn}
A subset $A$ of a topological space $(X, tau)$ is said to be \textbf{closed} if $X minus A$ is open in $X$, that is $X minus A is in tau$
\end{defn}

\begin{thm}
    \begin{itemize}
        \item[]
        \item  $empty set$ and $X$ are closed in $X$
        \item An intersection of any collection of closed sets is closed in $X$
        \item A union of a \textit{finite} collection of closed sets in $X$ is closed in $X$
    \end{itemize}
\end{thm}

\hhrule
