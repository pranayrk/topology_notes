\begin{defn}
    Let $X$ be a topological space. A collection $\beta$ of open subsets of $X$ is said to be a \textbf{basis} for the topology on $X$ if for every open set $U$ in $X$ and $x \in U$, there exists a $B \in \beta$ such that $x \in B \subseteq U$.

    Members of $\beta$ are called \textbf{basis open sets} corresponding to basis $\beta$.
\end{defn}

\begin{note}
    For any topological space $\left (X, \tau\right )$, $\tau$ is a basis for $\tau$.
\end{note}

\begin{note}
    In $\mathbb{R}$ the set $\left\{ \left (x - \varepsilon, x + \varepsilon\right )  :  x \in \mathbb{R}\text{ and }\varepsilon >0 \right\}$ is a basis for the usual topology.
\end{note}

\begin{note}
    For $\left (\mathbb{R}, \tau\right )$, where $\tau$ is the discrete topology, $\beta = \left\{ \left\{ x \right\}  :  x \in \mathbb{R} \right\}$ is a basis for $\tau$ on $\mathbb{R}$.
\end{note}

\begin{note}
    In $\mathbb{R}$ the set $\left\{ \left (x - \frac{1}{n}, x + \frac{1}{n}\right )  :  x \in \mathbb{R}\text{ and }n \in \mathbb{N} \right\}$ is a basis for the usual topology.
\end{note}

\begin{note}
    In $\mathbb{R}$ the set $\left\{ \left (a, b\right )  :  a,b \in \mathbb{Q}\text{ and }a < b \right\}$ is a basis for the usual topology. This is a countable basis for $\mathbb{R}$ with the usual topology.
\end{note}

\begin{note}
    Let $\left (X, \tau\right )$ be a metrizable space. Then $\beta = \left\{ B\left (x, \varepsilon\right )  :  x \in X, \varepsilon >0 \right\}$ is a basis for $\tau$.
\end{note}

\begin{defn}
    Let $\left (X, \tau\right )$ be a topological space. A collection $S \subseteq \tau$ is said to be a \textbf{subbasis} for the topology $\tau$ if for every open set $U$ in $\tau$ and $x \in U$, there exists a finite subcollection $\left\{ S_1, S_2, ..., S_n \right\}$ in $S$ such that $x \in \displaystyle\bigcap_{i = 1}^n S_i \subseteq U$

    Members of $S$ are called \textbf{subbasis open sets} corresponding to $S$.
\end{defn}

\begin{note}
    For a topological space $\left (X, \tau\right )$ a collection $S$ is a subbasis for the topology $\tau$ if and only if the collection of all finite intersections of members in $S$ forms a basis for the topology $\tau$.
\end{note}

\begin{thm}
    Let $\left (X, \tau\right )$ be a topological space and $\beta$ be a collection of open sets in $\left (X, \tau\right )$.

    Then $\beta$ is a basis for the topology $\tau$ on $X$ if and only if every open set $U$ in $\left (X, \tau\right )$ can be written as a union of members in $\beta$.
\end{thm}

\begin{thm}
    If $X$ is a set, a basis for a topology on $X$ is a collection $\beta$ of subsets of $X$ such that 
    \begin{itemize}
        \item for every $x \in X$ there exists $B \subset \beta$ containing $x$.
        \item if $x \in B_1 \cap B_2$, for some $B_1, B_2 \in \beta$, then there exists $B_3 \in \beta$ such that $x \in B_3 \subseteq B_1 \cap B_2$.
    \end{itemize}
\end{thm}

\begin{defn}
    The topology generated by the basis $\beta = \left\{ \left [a, b\right )  :  a, b \in \mathbb{R}, a < b \right\}$ is known as the \textbf{lower limit topology} on $\mathbb{R}$. 

    The space $\mathbb{R}$ equipped with the lower limit topology is known as the \textbf{Sorgenfrey line} $\mathbb{R}_l$.
\end{defn}

\begin{prop}
    The lower limit topology on $\mathbb{R}$ is strictly finer than the usual topology on $\mathbb{R}$
\end{prop}

\begin{defn}
    Let $K = \left\{ \frac{1}{n}  :  n \in \mathbb{N} \right\}$. Consider the basis \\ $\beta = \left\{ \left (a, b\right ) \setminus K  :  a,b \in \mathbb{Q}, a < b \right\} \cup \left\{ \left (a,b\right )  :  a,b \in \mathbb{R}, a < b \right\}$.  The topology generated by the basis $\beta$ is known as the \textbf{$k$-topology} on $\mathbb{R}$.
\end{defn}

\begin{prop}
    The $k$-topology on $\mathbb{R}$ is strictly finer than the usual topology on $\mathbb{R}$.
\end{prop}

\begin{prop}
    The lower limit topology and the $k$-topology on $\mathbb{R}$ are incomparable.
\end{prop}


\begin{defn}
    Let $X$ be a set. Let $S$ be a collection of subsets of $X$ whose union equals $X$. Any such collection is called a \textbf{subbasis} for a topology $X$.
\end{defn}

\begin{thm}
    Let $S$ be a subbasis for a topology on $X$. Then
    \\ $\tau = \{ U \subseteq X  :  $ for every $ x \in U $ there exists a finite number of members $ S_1, S_2, ..., S_n \in S $ such that $ x \in S_1 \cap S_2 \cap ... \cap S_n \subseteq U \}$ is the topology generated by the subbasis $S$.
\end{thm}

\begin{thm}
    Let $\beta$ be a basis for a topology on $X$. Then the topology generated by $\beta$ is equal to the intersection of all topologies on $X$ that contains $\beta$.
\end{thm}

\begin{thm}
    If $\beta$ is a basis for a topology on a set $X$, then the topology generated by $\beta$ on $\tau$ is the smallest topology on $X$ that contains $\beta$.
\end{thm}

\begin{defn}
    Let $(X, \tau)$ be a topological space.
    A collection $\mathscr{B}_x$ of neighbourhoods of $x$ is said to be a \textbf{basis at $\bm{x}$} if for every neighbourhood $U$ of $x$, there exists $B \in \mathscr{B}_x$ such that $x \in B \subseteq U$.
\end{defn}

\hhrule
