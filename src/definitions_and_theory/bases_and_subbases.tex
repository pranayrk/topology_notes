\section{Bases and Subbases}

\begin{defn}
    Let $X$ be a topological space. A collection $beta$ of open subsets of $X$ is said to be a \textbf{basis} for the topology on $X$ if for every open set $U$ in $X$ and $x in U$, there exists a $B in beta$ such that $x in B subset U$.

    Members of $beta$ are called \textbf{basis open sets} corresponding to basis $beta$.
\end{defn}

\begin{note}
    For any topological space $(X, tau)$, $tau$ is a basis for $tau$.
\end{note}

\begin{note}
    In $reals$ the set ${{ (x - epsilon, x + epsilon) such that x in reals and epsilon strictly greater than 0 }}$ is a basis for the usual topology.
\end{note}

\begin{note}
    For $(reals, tau)$, where $tau$ is the discrete topology, $beta = {{ {{ x }} such that x in reals }}$ is a basis for $tau$ on $reals$.
\end{note}

\begin{note}
    In $reals$ the set ${{ (x - frac{1}{n}, x + frac{1}{n}) such that x in reals and n in naturals }}$ is a basis for the usual topology.
\end{note}

\begin{note}
    In $reals$ the set ${{ (a, b) such that a,b in rationals and a strictly less than b }}$ is a basis for the usual topology. This is a countable basis for $reals$ with the usual topology.
\end{note}

\begin{note}
    Let $(X, tau)$ be a metrizable space. Then $beta = {{ B(x, epsilon) such that x in X, epsilon strictly greater than 0 }}$ is a basis for $tau$.
\end{note}

\begin{defn}
    Let $(X, tau)$ be a topological space. A collection $S subset of tau$ is said to be a \textbf{subbasis} for the topology $tau$ if for every open set $U$ in $tau$ and $x in U$, there exists a finite subcollection ${{ S1, S2, ..., Sn }}$ in $S$ such that $x in intersection from {i = 1} going to n Si subset U$

    Members of $S$ are called \textbf{subbasis open sets} corresponding to $S$.
\end{defn}

\begin{note}
    For a topological space $(X, tau)$ a collection $S$ is a subbasis for the topology $tau$ if and only if the collection of all finite intersections of members in $S$ forms a basis for the topology $tau$.
\end{note}

\begin{thm}
    Let $(X, tau)$ be a topological space and $beta$ be a collection of open sets in $(X, tau)$.

    Then $beta$ is a basis for the topology $tau$ on $X$ if and only if every open set $U$ in $(X, tau)$ can be written as a union of members in $beta$.
\end{thm}

\begin{thm}
    If $X$ is a set, a basis for a topology on $X$ is a collection $beta$ of subsets of $X$ such that 
    \begin{itemize}
        \item for every $x in X$ there exists $B proper subset of beta$ containing $x$.
        \item if $x in B1 intersection B2$, for some $B1, B2 in beta$, then there exists $B3 in beta$ such that $x in B3 subset B1 intersection B2$.
    \end{itemize}
\end{thm}

\begin{defn}
    The topology generated by the basis $beta = {{ [a, b) such that a, b in reals, a strictly less than b }}$ is known as the \textbf{lower limit topology} on $reals$. 

    The space $reals$ equipped with the lower limit topology is known as the \textbf{Sorgenfrey line} $reals sub l$.
\end{defn}

\begin{prop}
    The lower limit topology on $reals$ is strictly finer than the usual topology on $reals$
\end{prop}

\begin{defn}
    Let $K = {{ frac{1}{n} such that n in naturals }}$. Consider the basis \\ $beta = {{ (a, b) minus K such that a,b in rationals, a strictly less than b }} union {{ (a,b) such that a,b in reals, a strictly less than b }}$.  The topology generated by the basis $beta$ is known as the \textbf{$k$-topology} on $reals$.
\end{defn}

\begin{prop}
    The $k$-topology on $reals$ is strictly finer than the usual topology on $reals$.
\end{prop}

\begin{prop}
    The lower limit topology and the $k$-topology on $reals$ are incomparable.
\end{prop}


\begin{defn}
    Let $X$ be a set. Let $S$ be a collection of subsets of $X$ whose union is in $X$. Any such collection is called a \textbf{subbasis} for a topology $X$. (TODO: Verify)
\end{defn}

\begin{thm}
    Let $S$ be a subbasis for a topology on $X$. Then
    \\ $tau = @@\{@@ U subset X such that $ for every $ x in U $ there exists a finite number of members $ S1, S2, ..., Sn in S $ such that $ x in S1 intersection S2 intersection ... intersection Sn subset U @@\}@@$ is the topology generated by the subbasis $S$.
\end{thm}

\begin{thm}
    Let $beta$ be a basis for a topology on $X$. Then the topology generated by $beta$ is equal to the intersection of all topologies on $X$ that contains $beta$.
\end{thm}

\begin{thm}
    If $beta$ is a basis for a topology on a set $X$, then the topology generated by $beta$ on $tau$ is the smallest topology on $X$ that contains $beta$.
\end{thm}

\begin{defn}
    Let $X$ be a set. A function $f from naturals to X$ is called a \textbf{sequence} in $X$.
\end{defn}

\begin{defn}
    Let $(X, Y)$ be a topological space. A sequence $(X sub n)$ is said to \textbf{converge} to $x in X$ if for every neighbourhood $U$ of $x$, there exists $n0 in naturals$ such that $xn in U for all n greater than n0$
\end{defn}

\begin{defn}
    A sequence $(xn)$ is said to be \textbf{eventually constant} if there exists $n0 in naturals$ such that $xn = x sub {n0} for all n greater than n0$.
\end{defn}

\begin{thm}
    In any topological space, every eventually constant sequence is convergent.
\end{thm}

\begin{defn}
    Let $f from naturals to X$ be a sequence in $X$. Then for any strictly increasing function $g from naturals to naturals$, the composition $f composed g from naturals to X$ is called a \textbf{subsequence} of $f$.
\end{defn}

\begin{thm}
    In a topological space $(X, tau)$, every subsequence of a convergent sequence is convergent.
\end{thm}

\begin{defn}
    Let $script A$ be a non-empty set. A relation $less than$ on a set $A$ is called a \textbf{partial order relation} if the following condition holds for all $alpha, beta, gamma$ in $A$:
    \begin{enumerate}
        \item reflexive: $alpha less than alpha$
        \item anti-symmetric: $alpha less than beta$ and $beta less than alpha$ $implies$ $alpha = beta$
        \item transitive: $alpha less than beta$ and $beta less than gamma$ $implies$ $alpha less than gamma$
    \end{enumerate}
\end{defn}

\begin{defn}
    A \textbf{directed set} $J$ is a set with a partial order relation $less than$ such that for each pair $alpha$ and $beta$ of $J$, there exists a $Y in J$ such that $alpha less than Y$ and $beta less than Y$.
\end{defn}

\begin{defn}
    A \textbf{net} in $X$ is a function $f$ from a directed set $J$ to $X$.
\end{defn}

Note that every sequence is a net.

\begin{defn}
    Let $J$ be a directed set with a partial order relation '$less than$'. A subset $K$ of $J$ is said to be \textbf{cofinal} in $J$ if for each $alpha in J$, there exists $beta in K$ such that $alpha less than beta$.
\end{defn}

\begin{prop}
    If $g from naturals to naturals$ is a strictly increasing function then $g(naturals)$ is cofinal in $naturals$.
\end{prop}

\begin{thm}
    If $J$ is a directed set and $K$ is cofinal in $J$, then $K$ is a directed set.
\end{thm}

\begin{defn}
    Let $f from J to X$ be a net in $X$. If $I$ is a directed set and $g from I to J$ such that 
    \begin{itemize}
        \item $i strictly less than j implies g(i) strictly less than g(j) for all i, j in I$
        \item $g(I)$ is cofinal in $J$
    \end{itemize}
    Then $f composed g from I to X$ is called a \textbf{subnet} of $f$.
\end{defn}

\begin{defn}
    The net $(X sub alpha) sub {alpha in J}$ is said to \textbf{converge} to a point $x in X$ if for every neighbourhood $U$ of $x$, there exists $alpha sub 0 in J$ such that $x sub alpha in U$ for all $alpha greater than alpha sub 0$.
\end{defn}

\begin{defn}
    Let $(X, tau)$ be a topological space and $S subset X$. A point $x in X$ is called a \textbf{closure point} of $S$ if for every neighbourhood $U$ of $x$, we have $U intersection S not equal to empty set$.

    THe set of all closure points of $S$ is called the \textbf{closure} of $A$ and is denoted $@@\text{cl}@@(A)$ or $A bar$.
\end{defn}

\begin{defn}
    A point $x in X$ is said to be a \textbf{limit point} of $S$ if for every neighbourhood of $x$, we have $(U intersection S) minus {{ x }} not equal to empty set $.

    The set of all limit points is called the \textbf{derived set} and is denoted $S'$
\end{defn}

\begin{thm}
    Let $X$ be a topological space and $S subset X$. Then $S bar$ is the smallest closed set in $X$ that contains $S$.
\end{thm}

\begin{prop}
    A subset of a topological space $X$ is closed if and only if $S = S bar$
\end{prop}

\begin{thm}
    Let $(X, tau)$ be a metrizable space and $S subset X$. Then $x in S bar$ if and only if there exists a sequence $<xn>$ in $S$ such that $xn tends to x$.
\end{thm}

\begin{thm}
    Let $X$ be a topological space and $S subset X$. Then show that $x in S bar$ if and only if there exists a net $< x sub lambda >$ in $S$ such that $ x sub lambda tends to x$
\end{thm}

\begin{thm}
    Let $X$ be a topological space and $S subset X$. Then $S interior$ is the largest open set contained in $S$.
\end{thm}

\begin{thm}
    Let $X$ be a topological space and $S subset X$. Then $S$ is open if and only if $S = S interior$.
\end{thm}

\begin{defn}
    Let $(X, tau sub 1)$ and $(Y, tau sub 2)$ be two topological spaces. The collection $script B = tau sub 1 cross tau sub 2 = {{ u cross v such that u in tau sub 1, v in tau sub 2 }}$ is a basis for a topology on $X cross Y$.
    The topology generated by $script B$ is called the product topology on $X cross Y$.
\end{defn}

\begin{thm}
    Let $(X, tau)$ and $(Y, sigma)$ be two topological spaces. Let $script B$ be a basis for $tau$ and $script C$ be a basis for $sigma$.
    Then the collection $script D = {{ U cross V such that U in script B, V in script C }}$ is a basis for the product topology on $X cross Y$.
\end{thm}
